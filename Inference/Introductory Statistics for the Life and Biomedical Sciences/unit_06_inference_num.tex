% Options for packages loaded elsewhere
\PassOptionsToPackage{unicode}{hyperref}
\PassOptionsToPackage{hyphens}{url}
%
\documentclass[
  ignorenonframetext,
  aspectratio=169]{beamer}
\usepackage{pgfpages}
\setbeamertemplate{caption}[numbered]
\setbeamertemplate{caption label separator}{: }
\setbeamercolor{caption name}{fg=normal text.fg}
\beamertemplatenavigationsymbolsempty
% Prevent slide breaks in the middle of a paragraph
\widowpenalties 1 10000
\raggedbottom
\setbeamertemplate{part page}{
  \centering
  \begin{beamercolorbox}[sep=16pt,center]{part title}
    \usebeamerfont{part title}\insertpart\par
  \end{beamercolorbox}
}
\setbeamertemplate{section page}{
  \centering
  \begin{beamercolorbox}[sep=12pt,center]{part title}
    \usebeamerfont{section title}\insertsection\par
  \end{beamercolorbox}
}
\setbeamertemplate{subsection page}{
  \centering
  \begin{beamercolorbox}[sep=8pt,center]{part title}
    \usebeamerfont{subsection title}\insertsubsection\par
  \end{beamercolorbox}
}
\AtBeginPart{
  \frame{\partpage}
}
\AtBeginSection{
  \ifbibliography
  \else
    \frame{\sectionpage}
  \fi
}
\AtBeginSubsection{
  \frame{\subsectionpage}
}
\usepackage{amsmath,amssymb}
\usepackage{iftex}
\ifPDFTeX
  \usepackage[T1]{fontenc}
  \usepackage[utf8]{inputenc}
  \usepackage{textcomp} % provide euro and other symbols
\else % if luatex or xetex
  \usepackage{unicode-math} % this also loads fontspec
  \defaultfontfeatures{Scale=MatchLowercase}
  \defaultfontfeatures[\rmfamily]{Ligatures=TeX,Scale=1}
\fi
\usepackage{lmodern}
\ifPDFTeX\else
  % xetex/luatex font selection
\fi
% Use upquote if available, for straight quotes in verbatim environments
\IfFileExists{upquote.sty}{\usepackage{upquote}}{}
\IfFileExists{microtype.sty}{% use microtype if available
  \usepackage[]{microtype}
  \UseMicrotypeSet[protrusion]{basicmath} % disable protrusion for tt fonts
}{}
\makeatletter
\@ifundefined{KOMAClassName}{% if non-KOMA class
  \IfFileExists{parskip.sty}{%
    \usepackage{parskip}
  }{% else
    \setlength{\parindent}{0pt}
    \setlength{\parskip}{6pt plus 2pt minus 1pt}}
}{% if KOMA class
  \KOMAoptions{parskip=half}}
\makeatother
\usepackage{xcolor}
\newif\ifbibliography
\usepackage{color}
\usepackage{fancyvrb}
\newcommand{\VerbBar}{|}
\newcommand{\VERB}{\Verb[commandchars=\\\{\}]}
\DefineVerbatimEnvironment{Highlighting}{Verbatim}{commandchars=\\\{\}}
% Add ',fontsize=\small' for more characters per line
\usepackage{framed}
\definecolor{shadecolor}{RGB}{248,248,248}
\newenvironment{Shaded}{\begin{snugshade}}{\end{snugshade}}
\newcommand{\AlertTok}[1]{\textcolor[rgb]{0.94,0.16,0.16}{#1}}
\newcommand{\AnnotationTok}[1]{\textcolor[rgb]{0.56,0.35,0.01}{\textbf{\textit{#1}}}}
\newcommand{\AttributeTok}[1]{\textcolor[rgb]{0.13,0.29,0.53}{#1}}
\newcommand{\BaseNTok}[1]{\textcolor[rgb]{0.00,0.00,0.81}{#1}}
\newcommand{\BuiltInTok}[1]{#1}
\newcommand{\CharTok}[1]{\textcolor[rgb]{0.31,0.60,0.02}{#1}}
\newcommand{\CommentTok}[1]{\textcolor[rgb]{0.56,0.35,0.01}{\textit{#1}}}
\newcommand{\CommentVarTok}[1]{\textcolor[rgb]{0.56,0.35,0.01}{\textbf{\textit{#1}}}}
\newcommand{\ConstantTok}[1]{\textcolor[rgb]{0.56,0.35,0.01}{#1}}
\newcommand{\ControlFlowTok}[1]{\textcolor[rgb]{0.13,0.29,0.53}{\textbf{#1}}}
\newcommand{\DataTypeTok}[1]{\textcolor[rgb]{0.13,0.29,0.53}{#1}}
\newcommand{\DecValTok}[1]{\textcolor[rgb]{0.00,0.00,0.81}{#1}}
\newcommand{\DocumentationTok}[1]{\textcolor[rgb]{0.56,0.35,0.01}{\textbf{\textit{#1}}}}
\newcommand{\ErrorTok}[1]{\textcolor[rgb]{0.64,0.00,0.00}{\textbf{#1}}}
\newcommand{\ExtensionTok}[1]{#1}
\newcommand{\FloatTok}[1]{\textcolor[rgb]{0.00,0.00,0.81}{#1}}
\newcommand{\FunctionTok}[1]{\textcolor[rgb]{0.13,0.29,0.53}{\textbf{#1}}}
\newcommand{\ImportTok}[1]{#1}
\newcommand{\InformationTok}[1]{\textcolor[rgb]{0.56,0.35,0.01}{\textbf{\textit{#1}}}}
\newcommand{\KeywordTok}[1]{\textcolor[rgb]{0.13,0.29,0.53}{\textbf{#1}}}
\newcommand{\NormalTok}[1]{#1}
\newcommand{\OperatorTok}[1]{\textcolor[rgb]{0.81,0.36,0.00}{\textbf{#1}}}
\newcommand{\OtherTok}[1]{\textcolor[rgb]{0.56,0.35,0.01}{#1}}
\newcommand{\PreprocessorTok}[1]{\textcolor[rgb]{0.56,0.35,0.01}{\textit{#1}}}
\newcommand{\RegionMarkerTok}[1]{#1}
\newcommand{\SpecialCharTok}[1]{\textcolor[rgb]{0.81,0.36,0.00}{\textbf{#1}}}
\newcommand{\SpecialStringTok}[1]{\textcolor[rgb]{0.31,0.60,0.02}{#1}}
\newcommand{\StringTok}[1]{\textcolor[rgb]{0.31,0.60,0.02}{#1}}
\newcommand{\VariableTok}[1]{\textcolor[rgb]{0.00,0.00,0.00}{#1}}
\newcommand{\VerbatimStringTok}[1]{\textcolor[rgb]{0.31,0.60,0.02}{#1}}
\newcommand{\WarningTok}[1]{\textcolor[rgb]{0.56,0.35,0.01}{\textbf{\textit{#1}}}}
\usepackage{graphicx}
\makeatletter
\def\maxwidth{\ifdim\Gin@nat@width>\linewidth\linewidth\else\Gin@nat@width\fi}
\def\maxheight{\ifdim\Gin@nat@height>\textheight\textheight\else\Gin@nat@height\fi}
\makeatother
% Scale images if necessary, so that they will not overflow the page
% margins by default, and it is still possible to overwrite the defaults
% using explicit options in \includegraphics[width, height, ...]{}
\setkeys{Gin}{width=\maxwidth,height=\maxheight,keepaspectratio}
% Set default figure placement to htbp
\makeatletter
\def\fps@figure{htbp}
\makeatother
\setlength{\emergencystretch}{3em} % prevent overfull lines
\providecommand{\tightlist}{%
  \setlength{\itemsep}{0pt}\setlength{\parskip}{0pt}}
\setcounter{secnumdepth}{-\maxdimen} % remove section numbering
\usepackage{amsmath,verbatim}

\usepackage{multirow}
\usepackage{fancyvrb}
\usepackage{manfnt}
\usepackage[normalem]{ulem}

\usepackage{hyperref}
\hypersetup{
	colorlinks=true,
	linkcolor=blue,
	urlcolor=blue}

%\usepackage[colorlinks=true]{hyperref}

\mode<presentation>{\usetheme{Malmoe}}

%\synctex=1

\setbeamertemplate{headline}{}


\setbeamerfont{footline}{size=\scriptsize}
\setbeamerfont{frametitle}{shape=\scshape}
\setbeamertemplate{itemize item}[circle]
\setbeamertemplate{itemize subitem}{\scriptsize$\diamond$}
\setbeamercovered{transparent}

\setbeamertemplate{navigation symbols}{}
\setbeamertemplate{footline}[frame number]{} 


\definecolor{forest}{rgb}{0, .5, 0}
\definecolor{brick}{rgb}{.5, 0, 0}
\definecolor{darkgreen}{rgb}{0, .5, 0}
\definecolor{darkred}{rgb}{.7, .15, .15}
\definecolor{darkblue}{rgb}{0, 0, .5}
\definecolor{Green}{rgb}{0.2,1,0.2}


\newcommand{\R}{\textsf{R}}
\newcommand{\RStudio}{\textsl{R Studio}}

\usepackage[english]{babel}
%\usepackage{palatino}
\usepackage[T1]{fontenc}


% make all tt fonts bold to look more like Verbatim
\usepackage{lmodern}
\renewcommand\ttfamily{\usefont{T1}{lmtt}{m}{n}}

% Comment these out if you don't want a slide with just the
% part/section/subsection/subsubsection title:
\AtBeginPart{
  \let\insertpartnumber\relax
  \let\partname\relax
  \frame{\partpage}
}
\AtBeginSection{
  \let\insertsectionnumber\relax
  \let\sectionname\relax
  \frame{\sectionpage}
}
\AtBeginSubsection{
  \let\insertsubsectionnumber\relax
  \let\subsectionname\relax
  \frame{\subsectionpage}
}

\newcommand{\columnsbegin}{\begin{columns}}
	\newcommand{\columnsend}{\end{columns}}

\newenvironment{twocol}[4]{
\begin{columns}[c]
\column{#1\textwidth}
#3
\column{#2\textwidth}
#4
\end{columns}
}

\def\begincols{\begin{columns}}
\def\begincol{\begin{column}}
\def\endcol{\end{column}}
\def\endcols{\end{columns}}
\ifLuaTeX
  \usepackage{selnolig}  % disable illegal ligatures
\fi
\IfFileExists{bookmark.sty}{\usepackage{bookmark}}{\usepackage{hyperref}}
\IfFileExists{xurl.sty}{\usepackage{xurl}}{} % add URL line breaks if available
\urlstyle{same}
\hypersetup{
  pdftitle={Unit 6: Inference for Numerical Data},
  pdfauthor={Statistics S-100 Teaching Team},
  hidelinks,
  pdfcreator={LaTeX via pandoc}}

\title{Unit 6: Inference for Numerical Data}
\author{Statistics S-100 Teaching Team}
\date{Summer 2024}

\begin{document}
\frame{\titlepage}

\begin{frame}[allowframebreaks]
  \tableofcontents[hideallsubsections]
\end{frame}
\hypertarget{introduction}{%
\section{Introduction}\label{introduction}}

\begin{frame}{Comparing two population means}
\protect\hypertarget{comparing-two-population-means}{}
\small

Two-sample data can be paired or unpaired (independent).

\begin{itemize}
\item
  Paired measurements for each `participant' or study unit

  \begin{itemize}
  \item
    each observation can be logically matched to one other observation
    in the data
  \item
    e.g., scores on a standardized test before taking a prep course
    versus scores after the prep course
  \end{itemize}
\item
  Two independent sets of measurements

  \begin{itemize}
  \item
    observations cannot be matched on a one-to-one basis
  \item
    e.g., scores on a standardized test of students who did take a prep
    course versus scores of students who did not
  \end{itemize}
\end{itemize}

The nature of the data dictate which two-sample testing procedure is
appropriate: the two-sample test for paired data, or the two-sample test
for independent group data.
\end{frame}

\hypertarget{two-sample-test-for-paired-data}{%
\section{Two-sample test for paired
data}\label{two-sample-test-for-paired-data}}

\begin{frame}[fragile]{wetsuits and swimming velocity}
\protect\hypertarget{wetsuits-and-swimming-velocity}{}
\columnsbegin

\column{0.50\textwidth}

\footnotesize

Did a new wetsuit design allow for increased swim velocities during the
2000 Olympics?

\vspace{0.5cm}

A study was conducted to assess the effect of wearing a wetsuit on swim
velocity.

\begin{itemize}
\item
  12 competitive swimmers were asked to swim 1500m at maximal velocity,
  once wearing a wetsuit and once wearing a standard swimsuit.
\item
  Order of wetsuit versus swimsuit randomized.
\item
  Investigators recorded mean velocity (m/sec) for each trial.
\end{itemize}

\column{0.50\textwidth}

\scriptsize

\begin{Shaded}
\begin{Highlighting}[]
\FunctionTok{library}\NormalTok{(oibiostat)}
\FunctionTok{data}\NormalTok{(}\StringTok{"swim"}\NormalTok{)}
\NormalTok{swim\_new }\OtherTok{\textless{}{-}}\NormalTok{ swim}
\FunctionTok{colnames}\NormalTok{(swim\_new) }\OtherTok{\textless{}{-}} \FunctionTok{c}\NormalTok{(}\StringTok{"id"}\NormalTok{, }\StringTok{"ws\_vel"}\NormalTok{, }
                        \StringTok{"ss\_vel"}\NormalTok{, }\StringTok{"vel\_diff"}\NormalTok{)}
\NormalTok{swim\_new}
\end{Highlighting}
\end{Shaded}

\begin{verbatim}
##    id ws_vel ss_vel vel_diff
## 1   1   1.57   1.49     0.08
## 2   2   1.47   1.37     0.10
## 3   3   1.42   1.35     0.07
## 4   4   1.35   1.27     0.08
## 5   5   1.22   1.12     0.10
## 6   6   1.75   1.64     0.11
## 7   7   1.64   1.59     0.05
## 8   8   1.57   1.52     0.05
## 9   9   1.56   1.50     0.06
## 10 10   1.53   1.45     0.08
## 11 11   1.49   1.44     0.05
## 12 12   1.51   1.41     0.10
\end{verbatim}

\columnsend
\end{frame}

\begin{frame}{Idea behind inference for paired data}
\protect\hypertarget{idea-behind-inference-for-paired-data}{}
\small

The velocities are paired by swimmer: each swimmer has two velocity
measurements.

Suppose that for each swimmer \(i\), we have observations \(x_{i, WS}\)
and \(x_{i, SS}\).

\begin{itemize}
\item
  Let \(d_i\) be the difference between the measurements:
  \[d_i = x_{i, WS} - x_{i, SS}\]

  \begin{itemize}
  \item
    \(x_{i, WS}\) is the wetsuit velocity measurement for swimmer \(i\)
  \item
    \(x_{i, SS}\) is the swimsuit velocity measurement for swimmer \(i\)
  \end{itemize}
\item
  Base inference on \(\overline{d}\), the sample mean of the individual
  differences \(d_i\):

  \[\overline{d} = \frac{\sum d_i}{n}\]
\end{itemize}
\end{frame}

\begin{frame}{Inference for paired data}
\protect\hypertarget{inference-for-paired-data}{}
\small

Let \(\delta\) be the population mean of the difference in velocities
during a 1500m swim if all competitive swimmers recorded swim velocities
with each suit type.

The null and alternative hypotheses are

\begin{itemize}
\item
  \(H_0: \delta= 0\), the average difference in swim velocities between
  swimming with a wetsuit versus a swimsuit equals 0

  \begin{itemize}
  \tightlist
  \item
    i.e., wetsuits do not change swim velocities
  \end{itemize}
\item
  \(H_A: \delta \neq 0\), the average difference in swim velocities
  between swimming with a wetsuit versus a swimsuit is non-zero

  \begin{itemize}
  \tightlist
  \item
    i.e., wetsuits do change swim velocities
  \end{itemize}
\end{itemize}

We can also compute a 95\% confidence interval for \(\delta\).
\end{frame}

\begin{frame}{Inference for paired data \ldots}
\protect\hypertarget{inference-for-paired-data-1}{}
\columnsbegin

\column{0.50\textwidth}

\footnotesize

The formula for the test statistic is
\[ t = \dfrac{\overline{d} -\delta_0}{s_d /\sqrt{n}},\]

where \(\overline{d}\) is the mean of the differences, \(s_d\) is the
standard deviation of the differences, and \(n\) is the number of
differences (i.e., number of pairs).

\begin{itemize}
\item
  A paired \(t\)-test is essentially a one-sample test of difference
  values.
\item
  The \(p\)-value is calculated from a \(t\) distribution with
  \(df = n - 1\).
\end{itemize}

\column{0.50\textwidth}

\footnotesize

A 95\% confidence interval for paired data has the form

\[\overline{d} \pm \left( t^{\star} \times \dfrac{s_d}{\sqrt{n}} \right) ,\]

where \(t^{\star}\) is the point on a \emph{t} distribution with
\(df = n - 1\) that has area 0.025 to its right.

\columnsend
\end{frame}

\begin{frame}[fragile]{Letting \textsf{R} do the work}
\protect\hypertarget{letting-do-the-work}{}
\scriptsize

\begin{Shaded}
\begin{Highlighting}[]
\CommentTok{\#two{-}sample syntax}
\FunctionTok{t.test}\NormalTok{(swim}\SpecialCharTok{$}\NormalTok{wet.suit.velocity, swim}\SpecialCharTok{$}\NormalTok{swim.suit.velocity, }
       \AttributeTok{alternative =} \StringTok{"two.sided"}\NormalTok{, }\AttributeTok{paired =} \ConstantTok{TRUE}\NormalTok{)}
\end{Highlighting}
\end{Shaded}

\begin{verbatim}
## 
##  Paired t-test
## 
## data:  swim$wet.suit.velocity and swim$swim.suit.velocity
## t = 12.318, df = 11, p-value = 8.885e-08
## alternative hypothesis: true mean difference is not equal to 0
## 95 percent confidence interval:
##  0.06365244 0.09134756
## sample estimates:
## mean difference 
##          0.0775
\end{verbatim}

Note: \texttt{t.test(x, y, paired = TRUE)} returns results based on the
differences \texttt{x - y}.
\end{frame}

\begin{frame}[fragile]{Letting \textsf{R} do the work\ldots{}}
\protect\hypertarget{letting-do-the-work-1}{}
\scriptsize

\begin{Shaded}
\begin{Highlighting}[]
\CommentTok{\#one{-}sample syntax}
\FunctionTok{t.test}\NormalTok{(swim}\SpecialCharTok{$}\NormalTok{velocity.diff, }\AttributeTok{mu =} \DecValTok{0}\NormalTok{, }\AttributeTok{alternative =} \StringTok{"two.sided"}\NormalTok{)}
\end{Highlighting}
\end{Shaded}

\begin{verbatim}
## 
##  One Sample t-test
## 
## data:  swim$velocity.diff
## t = 12.318, df = 11, p-value = 8.885e-08
## alternative hypothesis: true mean is not equal to 0
## 95 percent confidence interval:
##  0.06365244 0.09134756
## sample estimates:
## mean of x 
##    0.0775
\end{verbatim}
\end{frame}

\hypertarget{two-sample-test-for-independent-group-data}{%
\section{Two-sample test for independent group
data}\label{two-sample-test-for-independent-group-data}}

\begin{frame}{COVID-19 vaccine acceptance}
\protect\hypertarget{covid-19-vaccine-acceptance}{}
\small

A study conducted in July
2020\footnote{This was before COVID-19 vaccines were available.}
investigated factors associated with self-reported willingness to
receive a COVID-19 vaccine. Researchers collected data from 1,971
participants via a 15-minute online survey that described 10
hypothetical vaccines.

\begin{itemize}
\item
  For example, one hypothetical vaccine was described as being 70\%
  effective, approved and licensed by the US Food and Drug
  Administration, developed in the United States, and endorsed by Vice
  President Joe Biden.
\item
  A vaccine acceptance score from 1 to 7 was calculated, representing
  the average response to the question ``How likely or unlikely would
  you be to get Vaccine X?'' for the 10 hypothetical vaccines.
\item
  A higher score represents greater willingness.
\end{itemize}

Does mean vaccine acceptance score differ between respondents who
described their political ideology as ``very liberal'' versus ``very
conservative''?
\end{frame}

\begin{frame}{COVID-19 vaccine acceptance\ldots{}}
\protect\hypertarget{covid-19-vaccine-acceptance-1}{}
\scriptsize

\includegraphics{unit_06_inference_num_files/figure-beamer/unnamed-chunk-6-1.pdf}
\end{frame}

\begin{frame}{Inference for comparing two means}
\protect\hypertarget{inference-for-comparing-two-means}{}
\small

The parameter of interest is \(\mu_{VL} - \mu_{VC}\), the difference in
population mean vaccine acceptance score for those identifying as ``very
liberal'' versus ``very conservative''.

The null and alternative hypotheses are

\begin{itemize}
\item
  \(H_0: \mu_{VL} = \mu_{VC}\), the population mean vaccine acceptance
  score is the same for those identifying as ``very liberal'' versus
  ``very conservative''.
\item
  \(H_A: \mu_{VL} \neq \mu_{VC}\), the population mean vaccine
  acceptance score differs between those identifying as ``very liberal''
  versus ``very conservative''.
\item
  Equivalently, \(H_0: \mu_{VL} - \mu_{VC} = 0\) and
  \(H_A: \mu_{VL} - \mu_{VC} \neq 0\).
\end{itemize}

In general, the hypotheses are written in terms of \(\mu_1\) and
\(\mu_2\).\footnote{The numerical labels are arbitrary, so it is best to explicitly specify which group is considered group 1 versus group 2.}

\begin{itemize}
\item
  The parameter of interest is \(\mu_1 - \mu_2\).
\item
  The point estimate is \(\overline{x}_1 - \overline{x}_2\).
\end{itemize}
\end{frame}

\begin{frame}{Inference for comparing two means\ldots{}}
\protect\hypertarget{inference-for-comparing-two-means-1}{}
\columnsbegin

\column{0.50\textwidth}

\footnotesize

The \(t\)-statistic is:

\[t =\dfrac{ (\overline{x}_{1} - \overline{x}_{2})- (\mu_1 - \mu_2)}
  {\sqrt{\dfrac{s_1^2}{n_1} + \dfrac{s_2^2}{n_2}}} \]

\vspace{0.5cm}

The degrees of freedom for the null distribution are different than for
the paired data setting.

\begin{itemize}
\item
  Use this approximation when doing the test by hand:
  \(df = \text{min}(n_1 - 1, n_2 - 1)\).
\item
  R uses a better approximation called the Satterthwaite approximation:
  \[df = \dfrac{\left[(s_1^2/n_1) + (s_2^2/n_2)\right]^2}{\left[(s_1^2/n_1)^2/(n_1 - 1) + (s_2^2/n_2)^2/(n_2 - 1)\right]}\]
\end{itemize}

\column{0.50\textwidth}

\footnotesize

The 95\% confidence interval for the difference in population means has
the form
\[( \overline{x}_{1} - \overline{x}_{2}) \pm \left( t^{\star} \times  \sqrt{\frac{s_{1}^{2}}{n_{1}}+\frac{s_{2}^{2}}{n_{2}}} \right), \]

where \(t^{\star}\) is the point on a \(t\) distribution that has area
0.025 to the right, with the same degrees of freedom as used for
calculating the \(p\)-value of the associated test.

\columnsend
\end{frame}

\begin{frame}[fragile]{Letting R do the work}
\protect\hypertarget{letting-r-do-the-work}{}
\small

The tilde syntax is used when one vector contains the numerical variable
and one vector contains the grouping variable.

\vspace{0.3cm}

\scriptsize

\begin{Shaded}
\begin{Highlighting}[]
\CommentTok{\#tilde syntax}
\FunctionTok{t.test}\NormalTok{(vax}\SpecialCharTok{$}\NormalTok{likely }\SpecialCharTok{\textasciitilde{}}\NormalTok{ vax}\SpecialCharTok{$}\NormalTok{ideology, }\AttributeTok{mu =} \DecValTok{0}\NormalTok{, }\AttributeTok{paired =} \ConstantTok{FALSE}\NormalTok{, }\AttributeTok{alternative =} \StringTok{"two.sided"}\NormalTok{)}
\end{Highlighting}
\end{Shaded}

\begin{verbatim}
## 
##  Welch Two Sample t-test
## 
## data:  vax$likely by vax$ideology
## t = 4.4389, df = 423.85, p-value = 1.154e-05
## alternative hypothesis: true difference in means between group very liberal and group very conservative is not equal to 0
## 95 percent confidence interval:
##  0.3873987 1.0031275
## sample estimates:
##      mean in group very liberal mean in group very conservative 
##                        5.267829                        4.572566
\end{verbatim}
\end{frame}

\begin{frame}[fragile]{Letting R do the work\ldots{}}
\protect\hypertarget{letting-r-do-the-work-1}{}
\small

The comma syntax is used when there are two vectors of numerical data.

\vspace{0.3cm}

\scriptsize

\begin{Shaded}
\begin{Highlighting}[]
\CommentTok{\#comma syntax}
\FunctionTok{t.test}\NormalTok{(vax}\SpecialCharTok{$}\NormalTok{likely[vax}\SpecialCharTok{$}\NormalTok{ideology }\SpecialCharTok{==} \StringTok{"very liberal"}\NormalTok{], }
\NormalTok{       vax}\SpecialCharTok{$}\NormalTok{likely[vax}\SpecialCharTok{$}\NormalTok{ideology }\SpecialCharTok{==} \StringTok{"very conservative"}\NormalTok{])}
\end{Highlighting}
\end{Shaded}

\begin{verbatim}
## 
##  Welch Two Sample t-test
## 
## data:  vax$likely[vax$ideology == "very liberal"] and vax$likely[vax$ideology == "very conservative"]
## t = 4.4389, df = 423.85, p-value = 1.154e-05
## alternative hypothesis: true difference in means is not equal to 0
## 95 percent confidence interval:
##  0.3873987 1.0031275
## sample estimates:
## mean of x mean of y 
##  5.267829  4.572566
\end{verbatim}
\end{frame}

\hypertarget{statistical-power-and-sample-size}{%
\section{Statistical power and sample
size}\label{statistical-power-and-sample-size}}

\begin{frame}{Outcomes and errors in testing}
\protect\hypertarget{outcomes-and-errors-in-testing}{}
\columnsbegin

\column{0.60\textwidth}

\footnotesize

Hypothesis tests can potentially result in incorrect decisions. The
following table shows the four possible ways that the conclusion of a
hypothesis test can be right or wrong.

\begin{center}
\begin{tabular}{|l|l|l|} \hline
& \multicolumn{2}{c|}{\textbf{Result of test}} \\ \hline
\textbf{State of nature}  & \textbf{Reject $H_0$}  & \textbf{Fail to reject $H_0$}\\
\hline
 & Type I error, & No error,\\
$H_0$ is true & probability = $\alpha$ & probability = $1 - \alpha$ \\
 & (false positive) & (true negative) \\ \hline
 &  No error, & Type II error,\\
$H_A$ is true &  probability = $1 - \beta$ & probability = $\beta$\\
 & (true positive) & (false negative) \\ \hline
\end{tabular}
\end{center}

\column{0.40\textwidth}

\footnotesize

In a trial, the defendant is either innocent or guilty. After hearing
evidence from both the prosecution and the defense, the court must reach
a verdict.

\vspace{0.5cm}

Under many legal systems, presumption of innocence is a legal right of
the accused.

\begin{itemize}
\tightlist
\item
  \(H_0\): The defendant is innocent.
\item
  \(H_A\): The defendant is guilty.
\end{itemize}

\vspace{0.5cm}

What does a Type I error represent in this context?

\vspace{0.25cm}

What does a Type II error represent in this context?

\columnsend
\end{frame}

\begin{frame}{Error probabilities in hypothesis testing}
\protect\hypertarget{error-probabilities-in-hypothesis-testing}{}
\small

Lab 2 uses simulation to explore how Type I and Type II error are
controlled.

\begin{itemize}
\tightlist
\item
  Type I error is controlled via rejecting \(H_0\) only when a
  \(p\)-value is smaller than \(\alpha\).
\end{itemize}

Statisticians are most often concerned with the complement of making a
Type II error.

\begin{itemize}
\item
  The \textbf{power} of a statistical test is the probability that the
  test will reject the null hypothesis when the alternative hypothesis
  is true.
\item
  In other words, power is the probability of correctly rejecting
  \(H_0\).
\end{itemize}
\end{frame}

\begin{frame}{Factors affecting power}
\protect\hypertarget{factors-affecting-power}{}
\small

The power of a statistical test is determined
by\footnote{The significance level $\alpha$ also influences power. More on this in Lab 2...}

\begin{itemize}
\item
  the hypothesized difference between two population means, also known
  as the population \textbf{effect size} (\(|\mu_1 - \mu_2|\))
\item
  the population standard deviation of each group
  (\(\sigma_1, \sigma_2\))
\item
  the sample size of each group (\(n_1, n_2\))
\end{itemize}

Think about power as a measure of how easy it is to distinguish whether
two groups have different (population) means.

Usually, a study team can only control sample size.
\end{frame}

\begin{frame}{Example: Treatment for Alzheimer's}
\protect\hypertarget{example-treatment-for-alzheimers}{}
\small

Treatment for Alzheimer's disease is an active research area. The
Dementia Severity Rating Scale (DSRS) measures impairment severity of a
person with Alzheimer's. Scores range from 0 (indicating no impairment)
to 54 (extreme impairment).

Cognitive decline over several years is measured by annual rate in
change in DSRS score.

\begin{itemize}
\tightlist
\item
  Example: a person observed for 3 consecutive years whose score
  increased from 7 to 14.5 has an annual rate of change of 7.5/3 = 2.5
  points per year.
\end{itemize}

Suppose a pharmaceutical company developed a drug to slow cognitive
decline in individuals affected by Alzheimer's disease.

\begin{itemize}
\item
  Conduct a randomized trial, comparing standard drug to new drug.
\item
  Enroll newly diagnosed patients, measure DSRS at beginning of study
  and 3 years later.
\item
  Compare average annual change in DSRS score between the two groups.
\end{itemize}

Test \(H_0: \mu_{ctrl} = \mu_{trt}\) against
\(H_A: \mu_{ctrl} \neq \mu_{trt}\).
\end{frame}

\begin{frame}{Intuition behind power}
\protect\hypertarget{intuition-behind-power}{}
\columnsbegin

\column{0.50\textwidth}

\footnotesize

\begin{itemize}
\item
  Suppose \(H_A\) is true and that we know the population distribution
  of DSRS score among the control group and treatment group:

  \begin{itemize}

    \footnotesize

    \item Control: $\mu_{ctrl} = 3.5$ pts/yr

    \item Treatment: $\mu_{trt} = 2.5$ pts/yr

    \end{itemize}
\item
  What would the sampling distributions of \(\overline{X}_{trt}\) and
  \(\overline{X}_{ctrl}\) look like?
\item
  If we only get to observe one random sample from each group, how
  likely would we be to conclude that the population means are
  different?

  \begin{itemize}

    \footnotesize

    \item i.e., How likely would we be to \textbf{reject $H_0$ correctly}?

    \end{itemize}
\end{itemize}

\column{0.50\textwidth}

\includegraphics{unit_06_inference_num_files/figure-beamer/unnamed-chunk-9-1.pdf}

\columnsend
\end{frame}

\begin{frame}{Intuition behind power: effect size}
\protect\hypertarget{intuition-behind-power-effect-size}{}
\columnsbegin

\column{0.50\textwidth}

\footnotesize

\begin{itemize}
\item
  \textbf{Effect size} (\(|\mu_1 - \mu_2|\)) refers to the hypothesized
  difference between two population means.
\item
  On the previous slide, we supposed that \(\mu_{ctrl}\) = 3.5 pts/yr
  and \(\mu_{trt} = 2.5\) pts/yr, which corresponds to effect size
  \(\Delta = |\mu_{ctrl} - \mu_{trt}| = |3.5 - 2.5|\) = 1 pt/yr.
\item
  What if the effect size were larger; i.e., if the new drug were
  actually more effective?

  \begin{itemize}

    \footnotesize

    \item What if $\mu_{trt} = 1.5$ pts/yr, $\Delta = 2$ pts/yr?

    \item What if $\mu_{trt} = 0.5$ pts/yr, $\Delta = 3$ pts/yr?

    \end{itemize}
\item
  As effect size increases, does power increase or decrease?

  \begin{itemize}

    \footnotesize

    \item As effect size increases, does it become easier or harder to detect a difference in population means?

    \end{itemize}
\end{itemize}

\column{0.50\textwidth}

\includegraphics{unit_06_inference_num_files/figure-beamer/effect_size-1.pdf}

\columnsend
\end{frame}

\begin{frame}{Intuition behind power: sample size}
\protect\hypertarget{intuition-behind-power-sample-size}{}
\columnsbegin

\column{0.50\textwidth}

\footnotesize

\begin{itemize}
\item
  How does \textbf{sample size} affect power?
\item
  Hint: recall that the standard error of \(\overline{X}\) depends on
  \(n\).
\item
  Let's run some simulations with different sample size.

  \begin{itemize}

    \footnotesize

    \item What if $n_{trt} = n_{ctrl} = 50$?

    \item What if $n_{trt} = n_{ctrl} = 100$?

    \item What if $n_{trt} = n_{ctrl} = 300$?

    \end{itemize}
\item
  As sample size increases, does power increase or decrease?

  \begin{itemize}

    \footnotesize

    \item As sample size increases, does it become easier or harder to detect a difference in population means?

    \end{itemize}
\end{itemize}

\column{0.50\textwidth}

\includegraphics{unit_06_inference_num_files/figure-beamer/sample_size-1.pdf}

\columnsend
\end{frame}

\begin{frame}{Intuition behind power: standard deviation}
\protect\hypertarget{intuition-behind-power-standard-deviation}{}
\columnsbegin

\column{0.50\textwidth}

\footnotesize

\begin{itemize}
\item
  How does \textbf{standard deviation} (\(\sigma_{trt}, \sigma_{ctrl}\))
  affect power?
\item
  Hint: recall that the standard error of \(\overline{X}\) depends on
  \(\sigma\).
\item
  Let's run some simulations with different standard deviation of DSRS
  score.

  \begin{itemize}

    \footnotesize

    \item What if $\sigma_{trt} = \sigma_{ctrl} = 4$?

    \item What if $\sigma_{trt} = \sigma_{ctrl} = 6$?

    \item What if $\sigma_{trt} = \sigma_{ctrl} = 10$?

    \end{itemize}
\item
  As standard deviation increases, does power increase or decrease?

  \begin{itemize}

    \footnotesize

    \item As standard deviation increases, does it become easier or harder to detect a difference in population means?

    \end{itemize}
\end{itemize}

\column{0.50\textwidth}

\includegraphics{unit_06_inference_num_files/figure-beamer/sigma-1.pdf}

\columnsend
\end{frame}

\begin{frame}{Background for power simulation lab\ldots{}}
\protect\hypertarget{background-for-power-simulation-lab}{}
\small

Part I: Controlling Type I error

\begin{itemize}
\item
  Assume that \(H_0: \mu_{ctrl} = \mu_{trt}\) is true.
\item
  Draw repeated samples from simulated control and treatment
  populations.
\item
  Conduct a hypothesis test for each set of two samples.
\item
  How many samples resulted in the \textbf{correct decision}?
\end{itemize}

Part II: Controlling Type II error

\begin{itemize}
\item
  Assume that \(H_A: \mu_{ctrl} \neq \mu_{trt}\) is true.
\item
  Draw repeated samples from simulated control and treatment
  populations.
\item
  Conduct a hypothesis test for each set of two samples.
\item
  How many samples resulted in the \textbf{correct decision}?

  \begin{itemize}
  \tightlist
  \item
    How does this number change when we change the sample size?
    \ldots the standard deviation? \ldots the effect size?
  \end{itemize}
\end{itemize}
\end{frame}

\begin{frame}{Choosing the right sample size}
\protect\hypertarget{choosing-the-right-sample-size}{}
\small

Study design often includes calculating a sample size such that the
probability of rejecting a null hypothesis correctly is acceptably
large, typically between 0.80 and 0.90.

It is important to have a precise estimate of an appropriate study size.

\begin{itemize}
\item
  A study needs to be large enough to allow for sufficient power to
  detect a difference between groups.
\item
  However, unnecessarily large studies are expensive, and can even be
  unethical.
\end{itemize}
\end{frame}

\begin{frame}{A typical sample size question}
\protect\hypertarget{a-typical-sample-size-question}{}
\small

A pharmaceutical company has developed a new drug to lower blood
pressure and is planning a clinical trial to test its effectiveness.

\begin{itemize}
\item
  Individuals whose systolic blood pressure is between 140 and 180 mmHg
  will be recruited.
\item
  Based on previous studies, blood pressures from such individuals will
  be approximately normally distributed with standard deviation of about
  12 mmHg.
\item
  Participants will be randomly assigned to the new drug or a standard
  drug, and the company will measure the difference in mean blood
  pressure levels between the groups.
\end{itemize}

The company expects to receive FDA approval for the drug if there is
evidence that the drug lowers blood pressure, on average, by at least 3
mmHg more than the standard drug.

How large should the study be if the company wants the power of the
study to be 0.80 (80\%)?
\end{frame}

\begin{frame}[fragile]{A typical sample size question \ldots}
\protect\hypertarget{a-typical-sample-size-question-1}{}
\small

We will use the \texttt{power.t.test()} function for power and sample
size
calculations.\footnote{\textit{OI Biostat} Section 5.4 has an extended discussion of this example, with formulas for hand calculations.}

To achieve a power of at least 0.80, the company should recruit at least
253 patients for each group; i.e., a total of at least 506 patients.

\scriptsize

\begin{Shaded}
\begin{Highlighting}[]
\FunctionTok{power.t.test}\NormalTok{(}\AttributeTok{n =} \ConstantTok{NULL}\NormalTok{, }\AttributeTok{delta =} \DecValTok{3}\NormalTok{, }\AttributeTok{sd =} \DecValTok{12}\NormalTok{, }\AttributeTok{sig.level =} \FloatTok{0.05}\NormalTok{, }\AttributeTok{power =} \FloatTok{0.80}\NormalTok{)}
\end{Highlighting}
\end{Shaded}

\begin{verbatim}
## 
##      Two-sample t test power calculation 
## 
##               n = 252.1281
##           delta = 3
##              sd = 12
##       sig.level = 0.05
##           power = 0.8
##     alternative = two.sided
## 
## NOTE: n is number in *each* group
\end{verbatim}
\end{frame}

\hypertarget{comparing-several-means-with-anova}{%
\section{Comparing several means with
ANOVA}\label{comparing-several-means-with-anova}}

\begin{frame}{Type I error rate for a single test}
\protect\hypertarget{type-i-error-rate-for-a-single-test}{}
\small

Suppose we are interested in comparing means across more than two
groups. Why not conduct several two-sample \(t\)-tests?

\begin{itemize}
\item
  If there are \(k\) groups, then \({k \choose 2} = \frac{k(k-1)}{2}\)
  \(t\)-tests are needed.
\item
  Conducting multiple tests on the same data increases the overall rate
  of Type I error.
\end{itemize}

Recall that making a Type I error (rejecting \(H_0\) when \(H_0\) is
true) occurs with probability \(\alpha\).

\begin{itemize}
\item
  Type I error rate is controlled by rejecting only when a test
  \(p\)-value is smaller than \(\alpha\).
\item
  \(\alpha\) is typically kept low.
\item
  With a single two-group comparison at \(\alpha = 0.05\), there is a
  5\% chance of incorrectly identifying an association where none
  actually exists.
\end{itemize}
\end{frame}

\begin{frame}{What about several tests?}
\protect\hypertarget{what-about-several-tests}{}
\small

What happens to Type I error when making several comparisons?

When conducting more than one \(t\)-test in an analysis\ldots{}

\begin{itemize}
\item
  The significance level (\(\alpha\)) used in each test controls the
  error rate for that test.
\item
  The \textbf{experiment-wise error rate} is the chance that at least
  one test will incorrectly reject \(H_0\) when all tested null
  hypotheses are true.
\item
  Controlling experiment-wise error rate is one specific approach for
  controlling Type I error.
\end{itemize}
\end{frame}

\begin{frame}{Probability of experiment-wise error}
\protect\hypertarget{probability-of-experiment-wise-error}{}
\small

Suppose that two \(t\)-tests are conducted to assess whether two new
drug candidates are associated with better outcome. Assume the tests are
independent and each are conducted at the \(\alpha = 0.05\) significance
level.

Let \(A\) be the event of making a Type I error on the first test, and
\(B\) be the event of making a Type I error on the second test, where
\(P(A) = P(B) = 0.05\).

The probability of making at least one error is equal to the complement
of the event that a Type I error is not made with either test.
\[1 - [P(A^C) P(B^C) ] = 1 - (1 - 0.05)^2 = 0.0975 \]

Thus, when making two independent \(t\)-tests, there is about a 10\%
chance of making at least one Type I error; the experiment-wise error is
10\%.
\end{frame}

\begin{frame}{Probability of experiment-wise error\ldots{}}
\protect\hypertarget{probability-of-experiment-wise-error-1}{}
\small

With 10 tests\ldots{}
\[\text{experiment-wise error }=  1 - (1 - 0.05)^{10} = 0.401\]

With 25 tests\ldots{}
\[\text{experiment-wise error }=  1 - (1 - 0.05)^{25} = 0.723\]

With 100 tests\ldots{}
\[\text{experiment-wise error }=  1 - (1 - 0.05)^{100} = 0.994\]

With 100 independent tests, there is a 99.4\% chance an investigator
will make at least one Type I error!

\begin{itemize}
\tightlist
\item
  If a company tested 100 drug candidates (that actually aren't
  effective), there is a 99.4\% chance of incorrectly concluding that at
  least one candidate is effective.
\end{itemize}
\end{frame}

\begin{frame}{Analysis of Variance (ANOVA)}
\protect\hypertarget{analysis-of-variance-anova}{}
\small

ANOVA uses a single hypothesis test to assess whether means across
several groups are equal:

\begin{itemize}
\item
  \(H_0\): mean outcome is same across all groups
  (\(\mu_1 = \mu_2 = \mu_3 = ... = \mu_k\))
\item
  \(H_A\): at least one mean is different from the others (i.e., means
  are not all equal)
\end{itemize}

By using a single test, the Type I error rate is still maintained at
\(\alpha\).
\end{frame}

\begin{frame}{Idea behind ANOVA}
\protect\hypertarget{idea-behind-anova}{}
\small

Are the sample means different enough that \(H_0\) seems unlikely to be
true?

\begin{itemize}
\tightlist
\item
  i.e.~Does it seem like the sample means come from different
  populations, or could it be plausible that they are simply different
  samples drawn from one population?
\end{itemize}

Compare two quantities:

\begin{itemize}
\item
  Variability between groups (\(MSG\)): how different are the group
  means from each other, i.e., how much does each group mean vary from
  the overall mean?
\item
  Variability within groups (\(MSE\)): how variable are the data within
  each group?
\end{itemize}

\footnotesize

\(MSG\) denotes mean square between groups, while \(MSE\) denotes mean
square error. Refer to \textit{OI Biostat} Section 5.5.1 for details.
\end{frame}

\begin{frame}{Idea behind ANOVA\ldots{}}
\protect\hypertarget{idea-behind-anova-1}{}
\begin{figure}[]
\includegraphics[height = 5cm]
{figures/toyANOVA.pdf}
\end{figure}

\small

\begin{itemize}
\tightlist
\item
  I, II, and III: difficult to discern differences in means, variability
  within each group is high
\item
  IV, V, and VI: appears to be differences in means, these differences
  are large relative to variance within each group
\end{itemize}
\end{frame}

\begin{frame}{Idea behind ANOVA\ldots{}}
\protect\hypertarget{idea-behind-anova-2}{}
\footnotesize

Under the null hypothesis, there is no real difference between the
groups; thus, any observed variation in group means is due to chance.

\begin{itemize}
\item
  Think of all observations as belonging to a single group.
\item
  Variability between group means should equal variability within
  groups.
\end{itemize}

The \textit{F-statistic} is the test statistic for ANOVA.
\[F = \frac{\text{variance between groups}}{\text{variance within groups}}  = \frac{MSG}{MSE}\]

\begin{itemize}
\item
  If the \(F\)-statistic is approximately 1, this suggests that the null
  hypothesis is true.
\item
  If the \(F\)-statistic is (much) larger than 1, this suggests that the
  population means are different.
\item
  The \(F\)-statistic follows an \(F\) distribution, with two degrees of
  freedom, \(df_1\) and \(df_2\); \(df_1 = n_{groups} - 1\),
  \(df_2 = n_{obs} - n_{groups}\).
\item
  The \(p\)-value for ANOVA equals \(P(F \geq F\text{-stat})\).
\end{itemize}
\end{frame}

\begin{frame}{Assumptions for ANOVA}
\protect\hypertarget{assumptions-for-anova}{}
\small

Important to check whether the assumptions for conducting ANOVA are
reasonably satisfied:

\begin{enumerate}
\item
  Observations are independent within and across groups.

  \begin{itemize}
  \tightlist
  \item
    Think about study design/context.
  \end{itemize}
\item
  Data within each group are approximately normal.

  \begin{itemize}
  \item
    Look at the data graphically, such as with a histogram.
  \item
    Normal Q-Q plots can help\ldots{}
  \end{itemize}
\item
  Variability across groups is about equal.

  \begin{itemize}
  \item
    Look at the data graphically.
  \item
    Numerical rule of thumb: ratio of largest variance to smallest
    variance \textless{} 3 is considered ``about equal''.
  \end{itemize}
\end{enumerate}
\end{frame}

\begin{frame}{Normal probability plots (Q-Q plots)}
\protect\hypertarget{normal-probability-plots-q-q-plots}{}
\footnotesize

\begin{figure}[]
\includegraphics[height = 5cm]
{figures/normalExamples.pdf}
\end{figure}

If points fall on or near the line, data closely follow a normal
distribution.

\begin{itemize}
\item
  Difficult to evaluate in small datasets.
\item
  Plots show three simulated normal datasets: from L to R, \(n = 40\),
  \(n = 100\), \(n = 400\)
\end{itemize}
\end{frame}

\begin{frame}{Normal probability plots (Q-Q plots)\ldots{}}
\protect\hypertarget{normal-probability-plots-q-q-plots-1}{}
\scriptsize

\includegraphics{unit_06_inference_num_files/figure-beamer/unnamed-chunk-17-1.pdf}
\end{frame}

\begin{frame}{Pairwise comparisons}
\protect\hypertarget{pairwise-comparisons}{}
\small

If the \(F\)-test indicates there is sufficient evidence that the group
means are not all equal, proceed with pairwise comparisons to identify
which group means are different.

Pairwise comparisons are made using the two-sample \(t\)-test for
independent groups.

\begin{itemize}
\item
  To maintain the overall Type I error rate at \(\alpha\), each pairwise
  comparison is conducted at at an adjusted significance level referred
  to as \(\alpha^\star\).
\item
  The Bonferroni correction is one method for adjusting \(\alpha\).
  \[\alpha^\star = \alpha/K, \text{ where } K = \frac{k(k-1)}{2} \text{ for $k$ groups}\]
\item
  Note that the Bonferroni correction is a very stringent (i.e.,
  conservative) correction, made under the assumption that all tests are
  independent.
\end{itemize}
\end{frame}

\begin{frame}{COVID-19 vaccine acceptance, again}
\protect\hypertarget{covid-19-vaccine-acceptance-again}{}
\columnsbegin

\column{0.50\textwidth}

\footnotesize

With ANOVA, we can address a different version of the question addressed
in the two-group context.

\begin{itemize}
\item
  Does mean vaccine acceptance score differ by political party?
\item
  Respondents reported political party as either Republican, Democrat,
  or Independent.
\end{itemize}

\column{0.50\textwidth}

\scriptsize

\includegraphics{unit_06_inference_num_files/figure-beamer/unnamed-chunk-20-1.pdf}

\columnsend
\end{frame}

\begin{frame}[fragile]{COVID-19 vaccine acceptance, again\ldots{}}
\protect\hypertarget{covid-19-vaccine-acceptance-again-1}{}
\columnsbegin

\column{0.40\textwidth}

\footnotesize

Check assumptions:

\begin{enumerate}
\item
  \emph{Independence}. Data are from participants taking an online
  survey, so it is reasonable to consider the responses as independent
\item
  \emph{Approximate normality}. Q-Q plots indicate some departures from
  normality for the extreme values, but observations mostly follow
  normality in the centers
\item
  \emph{Approximately constant variance}. ratio of largest to smallest
  variance is 1.65, which is considered ``about equal''
\end{enumerate}

\column{0.60\textwidth}

\scriptsize

\includegraphics{unit_06_inference_num_files/figure-beamer/unnamed-chunk-21-1.pdf}

\begin{Shaded}
\begin{Highlighting}[]
\CommentTok{\#check variance}
\FunctionTok{tapply}\NormalTok{(vax}\SpecialCharTok{$}\NormalTok{likely, vax}\SpecialCharTok{$}\NormalTok{party3, var)}
\end{Highlighting}
\end{Shaded}

\begin{verbatim}
##  republican    democrat independent 
##    3.439048    2.163715    2.083126
\end{verbatim}

\begin{Shaded}
\begin{Highlighting}[]
\FloatTok{3.44}\SpecialCharTok{/}\FloatTok{2.08}
\end{Highlighting}
\end{Shaded}

\begin{verbatim}
## [1] 1.653846
\end{verbatim}

\columnsend
\end{frame}

\begin{frame}[fragile]{COVID-19 vaccine acceptance, again\ldots{}}
\protect\hypertarget{covid-19-vaccine-acceptance-again-2}{}
\columnsbegin

\column{0.35\textwidth}

\footnotesize

State hypotheses.

\begin{itemize}
\item
  \(H_0: \mu_I = \mu_D = \mu_R\), mean vaccine acceptance score is equal
  across political party.
\item
  \(H_A\): at least one group has a mean vaccine acceptance score
  different from that of the other groups.
\end{itemize}

\vspace{0.5cm}

Let \(\alpha = 0.05\).

\column{0.65\textwidth}

\scriptsize

\begin{Shaded}
\begin{Highlighting}[]
\FunctionTok{summary}\NormalTok{(}\FunctionTok{aov}\NormalTok{(vax}\SpecialCharTok{$}\NormalTok{likely }\SpecialCharTok{\textasciitilde{}}\NormalTok{ vax}\SpecialCharTok{$}\NormalTok{party3))}
\end{Highlighting}
\end{Shaded}

\begin{verbatim}
##              Df Sum Sq Mean Sq F value Pr(>F)  
## vax$party3    2   23.7  11.864   4.188 0.0158 *
## Residuals   459 1300.4   2.833                 
## ---
## Signif. codes:  0 '***' 0.001 '**' 0.01 '*' 0.05 '.' 0.1 ' ' 1
\end{verbatim}

\footnotesize

\begin{itemize}
\item
  The \(p\)-value from the ANOVA \(F\)-test is 0.0158.
\item
  If the population mean vaccine acceptance scores were actually the
  same across groups, the probability of observing an \(F\)-statistic as
  large as 4.19 or larger equals 0.0158.
\item
  These data suggest that at least one population mean vaccine
  acceptance score is different from the others.
\end{itemize}

\columnsend
\end{frame}

\begin{frame}{Controlling Type I error rate}
\protect\hypertarget{controlling-type-i-error-rate}{}
\small

If the ANOVA \(F\)-test is significant, then it is appropriate to
proceed to conducting pairwise comparisons; i.e., using two-sample
\(t\)-tests to compare each possible pairing of the
groups.\footnote{These $t$-tests are typically referred to as \textit{post hoc} tests.}

\begin{itemize}
\item
  Each test should be conducted at the \(\alpha^\star\) significance
  level so that the overall Type I error rate remains at \(\alpha\).
\item
  These tests are still conducted under the assumption that the variance
  between groups is equal; thus, the test statistics are calculated
  using the pooled estimate of standard deviation between groups.
  Details are in \emph{OI Biostat} Section 5.5.3.
\item
  We will use \texttt{pairwise.t.test( )} to perform these \emph{post
  hoc} two-sample \emph{t}-tests.
\end{itemize}
\end{frame}

\begin{frame}[fragile]{Post hoc testing}
\protect\hypertarget{post-hoc-testing}{}
\small

Each test should be conducted at the \(\alpha^\star = 0.05/3 = 0.0167\)
significance level since there are 3 comparisons being made.

\begin{itemize}
\item
  There is evidence that Independents are different from both
  Republicans and Democrats.
\item
  There is not evidence that Republicans and Democrats are different.
\end{itemize}

\scriptsize

\begin{Shaded}
\begin{Highlighting}[]
\FunctionTok{pairwise.t.test}\NormalTok{(vax}\SpecialCharTok{$}\NormalTok{likely, vax}\SpecialCharTok{$}\NormalTok{party3, }\AttributeTok{p.adj =} \StringTok{"none"}\NormalTok{)}
\end{Highlighting}
\end{Shaded}

\begin{verbatim}
## 
##  Pairwise comparisons using t tests with pooled SD 
## 
## data:  vax$likely and vax$party3 
## 
##             republican democrat
## democrat    0.7424     -       
## independent 0.0046     0.0149  
## 
## P value adjustment method: none
\end{verbatim}
\end{frame}

\begin{frame}[fragile]{Letting \textsf{R} do the work\ldots{}}
\protect\hypertarget{letting-do-the-work-2}{}
\small

Setting \texttt{p.adj} to \texttt{"bonf"} instructs R to rescale the
\(p\)-values (by multiplying by \(K\), the total number of comparisons)
so they can be compared to the original \(\alpha\) level of 0.05.

\vspace{0.3cm}

\scriptsize

\begin{Shaded}
\begin{Highlighting}[]
\FunctionTok{pairwise.t.test}\NormalTok{(vax}\SpecialCharTok{$}\NormalTok{likely, vax}\SpecialCharTok{$}\NormalTok{party3, }\AttributeTok{p.adj =} \StringTok{"bonf"}\NormalTok{)}
\end{Highlighting}
\end{Shaded}

\begin{verbatim}
## 
##  Pairwise comparisons using t tests with pooled SD 
## 
## data:  vax$likely and vax$party3 
## 
##             republican democrat
## democrat    1.000      -       
## independent 0.014      0.045   
## 
## P value adjustment method: bonferroni
\end{verbatim}
\end{frame}

\hypertarget{a-closer-look-at-the-p-value}{%
\section{\texorpdfstring{A closer look at the
\(p\)-value}{A closer look at the p-value}}\label{a-closer-look-at-the-p-value}}

\begin{frame}{Fisher's \(p\)-value}
\protect\hypertarget{fishers-p-value}{}
\small

In 1925, Ronald Fisher published \emph{Statistical Methods for Research
Workers}, a book aimed at providing biologists with the means of
applying statistical tests accurately to numerical data.

\begin{itemize}
\item
  ``The value for which \(P = 0.05\), or 1 in 20, is 1.96 or nearly 2;
  it is convenient to take this point as a limit in judging whether a
  deviation is to be considered significant or not. Deviations exceeding
  twice the standard deviation are thus formally regarded as
  significant\ldots{}''
\item
  Provided a valuable rule of thumb at a time when the use of tables and
  approximations were essential for practicing researchers
\end{itemize}

Over time, the \(p\)-value in conjunction with the hypothesis testing
framework developed by Jerzy Neyman and Egon Pearson became widely used
across many scientific disciplines.
\end{frame}

\begin{frame}{The mis-use of \(p\)-values}
\protect\hypertarget{the-mis-use-of-p-values}{}
\columnsbegin

\column{0.70\textwidth}

\footnotesize

``\(P\)-hacking'' refers to the conscious or unconscious manipulation of
data in order to achieve a desired result (i.e., a significant
\(p\)-value):

\begin{itemize}
\item
  There are many possible decisions in the data analysis process, and no
  obviously correct way to proceed.
\item
  Unconscious bias can lead researchers to make decisions that will
  confirm what they would like to believe.
\item
  Non-significant results are difficult to publish in scientific
  journals.
\item
  Tempting to perform many hypothesis tests and only report the
  statistically significant results.
\end{itemize}

The following slide shows an interactive ``data analysis'' from
\href{https://fivethirtyeight.com/features/science-isnt-broken/}{"Science isn't Broken"}.\footnote{FiveThirtyEight, 19 Aug 2015.}

\column{0.30\textwidth}

\begin{figure}[]
\includegraphics[height = 7cm]
{figures/xkcdJellybeans.png}
\end{figure}

\columnsend
\end{frame}

\begin{frame}{The mis-use of \(p\)-values\ldots{}}
\protect\hypertarget{the-mis-use-of-p-values-1}{}
\begin{figure}[]
\includegraphics[height = 7cm]
{figures/pHacking.png}
\end{figure}
\end{frame}

\begin{frame}{Statistical significance versus practical significance}
\protect\hypertarget{statistical-significance-versus-practical-significance}{}
\columnsbegin

\column{0.45\textwidth}

\footnotesize

In 2016, the American Statistical Association released a formal
statement clarifying principles about the proper use and interpretation
of the \(p\)-value. Some main points mentioned:

\begin{itemize}
\item
  A \(p\)-value does not measure the size of an effect or the importance
  of a result. Statistical significance is not equivalent to scientific,
  human, or economic significance.
\item
  Scientific conclusions and business or policy decisions should not be
  based only on whether a \(p\)-value passes a specific threshold. Many
  other contextual factors should be considered, such as the design of a
  study, the external evidence for the phenomenon under study, and the
  validity of the assumptions that underlie the data analysis.
\end{itemize}

\column{0.55\textwidth}

\footnotesize

A 2013
\href{https://www.pnas.org/doi/10.1073/pnas.1222447110}{study published in \textit{PNAS}}
of 19,000+ people found that people who meet their spouses
online\ldots{}

\begin{itemize}
\item
  Are less likely to divorce than those who meet offline (\(p < 0.002\))
\item
  Are more likely to have higher marital satisfaction (\(p < 0.001\))
\end{itemize}

\vspace{0.25cm}

Important to examine the effect sizes:

\begin{itemize}
\item
  Divorce rate of 5.96\% for those who met online versus 7.67\% for
  those who met in person.
\item
  On a 7 point scale, happiness value of 5.64 for those who met online
  versus 5.48 for those who met offline.
\end{itemize}

Do these results provide compelling evidence that one should change
their
behavior?\footnote{\scriptsize Study results reported in a \textit{Scientific American} article: "Meeting Your Spouse Online May Lead to a Better Marriage".}

\columnsend
\end{frame}

\begin{frame}{Modern solutions}
\protect\hypertarget{modern-solutions}{}
\small

Emphasizing reproducibility and replicability.

\begin{itemize}
\item
  Making data and analysis code public.
\item
  Publishing negative/inconclusive results.
\end{itemize}

Reporting analysis plan before conducting experiments.

\begin{itemize}
\tightlist
\item
  e.g., a two-stage peer review process in which methods and proposed
  analyses are peer-reviewed prior to data collection and analysis
\end{itemize}

Focusing interpretation of data on effect sizes and confidence
intervals.

\begin{itemize}
\tightlist
\item
  Asking ``How much of an effect is there?'' versus ``Is there an
  effect?''
\end{itemize}

\href{https://www.tandfonline.com/doi/full/10.1080/00031305.2019.1583913}{Moving to a World Beyond "$p < 0.05$"}

\begin{itemize}
\tightlist
\item
  Editorial accompanying a special issue of \emph{The American
  Statistician} from Mar 2019, containing 43 papers discussing how to
  move to ``a world where researchers are free to treat `p = 0.051' and
  `p = 0.049' as not being categorically different.''
\end{itemize}
\end{frame}

\begin{frame}{What should you keep in mind?}
\protect\hypertarget{what-should-you-keep-in-mind}{}
\small

When conducting research\ldots{}

\begin{itemize}
\item
  Clearly specify details of an analysis approach (e.g., primary
  outcome, significance level, number of tests) \emph{before} viewing
  data.
\item
  Understand how to appropriately use and interpret \(p\)-values.
\item
  Report interval estimates, not just \(p\)-values.
\item
  When possible, replicate an analysis with new data.
\end{itemize}

When consuming research\ldots{}

\begin{itemize}
\item
  Be wary of strong conclusions based solely on a \(p\)-value.
\item
  Consider factors such as study design, measurement quality, validity
  of any assumptions made, and plausibility (e.g., evidence for a
  biological mechanism).
\item
  Avoid confusing statistical significance with practical significance.
\end{itemize}
\end{frame}

\end{document}
