% Options for packages loaded elsewhere
\PassOptionsToPackage{unicode}{hyperref}
\PassOptionsToPackage{hyphens}{url}
%
\documentclass[
  ignorenonframetext,
  aspectratio=169]{beamer}
\usepackage{pgfpages}
\setbeamertemplate{caption}[numbered]
\setbeamertemplate{caption label separator}{: }
\setbeamercolor{caption name}{fg=normal text.fg}
\beamertemplatenavigationsymbolsempty
% Prevent slide breaks in the middle of a paragraph
\widowpenalties 1 10000
\raggedbottom
\setbeamertemplate{part page}{
  \centering
  \begin{beamercolorbox}[sep=16pt,center]{part title}
    \usebeamerfont{part title}\insertpart\par
  \end{beamercolorbox}
}
\setbeamertemplate{section page}{
  \centering
  \begin{beamercolorbox}[sep=12pt,center]{part title}
    \usebeamerfont{section title}\insertsection\par
  \end{beamercolorbox}
}
\setbeamertemplate{subsection page}{
  \centering
  \begin{beamercolorbox}[sep=8pt,center]{part title}
    \usebeamerfont{subsection title}\insertsubsection\par
  \end{beamercolorbox}
}
\AtBeginPart{
  \frame{\partpage}
}
\AtBeginSection{
  \ifbibliography
  \else
    \frame{\sectionpage}
  \fi
}
\AtBeginSubsection{
  \frame{\subsectionpage}
}
\usepackage{amsmath,amssymb}
\usepackage{iftex}
\ifPDFTeX
  \usepackage[T1]{fontenc}
  \usepackage[utf8]{inputenc}
  \usepackage{textcomp} % provide euro and other symbols
\else % if luatex or xetex
  \usepackage{unicode-math} % this also loads fontspec
  \defaultfontfeatures{Scale=MatchLowercase}
  \defaultfontfeatures[\rmfamily]{Ligatures=TeX,Scale=1}
\fi
\usepackage{lmodern}
\ifPDFTeX\else
  % xetex/luatex font selection
\fi
% Use upquote if available, for straight quotes in verbatim environments
\IfFileExists{upquote.sty}{\usepackage{upquote}}{}
\IfFileExists{microtype.sty}{% use microtype if available
  \usepackage[]{microtype}
  \UseMicrotypeSet[protrusion]{basicmath} % disable protrusion for tt fonts
}{}
\makeatletter
\@ifundefined{KOMAClassName}{% if non-KOMA class
  \IfFileExists{parskip.sty}{%
    \usepackage{parskip}
  }{% else
    \setlength{\parindent}{0pt}
    \setlength{\parskip}{6pt plus 2pt minus 1pt}}
}{% if KOMA class
  \KOMAoptions{parskip=half}}
\makeatother
\usepackage{xcolor}
\newif\ifbibliography
\usepackage{color}
\usepackage{fancyvrb}
\newcommand{\VerbBar}{|}
\newcommand{\VERB}{\Verb[commandchars=\\\{\}]}
\DefineVerbatimEnvironment{Highlighting}{Verbatim}{commandchars=\\\{\}}
% Add ',fontsize=\small' for more characters per line
\usepackage{framed}
\definecolor{shadecolor}{RGB}{248,248,248}
\newenvironment{Shaded}{\begin{snugshade}}{\end{snugshade}}
\newcommand{\AlertTok}[1]{\textcolor[rgb]{0.94,0.16,0.16}{#1}}
\newcommand{\AnnotationTok}[1]{\textcolor[rgb]{0.56,0.35,0.01}{\textbf{\textit{#1}}}}
\newcommand{\AttributeTok}[1]{\textcolor[rgb]{0.13,0.29,0.53}{#1}}
\newcommand{\BaseNTok}[1]{\textcolor[rgb]{0.00,0.00,0.81}{#1}}
\newcommand{\BuiltInTok}[1]{#1}
\newcommand{\CharTok}[1]{\textcolor[rgb]{0.31,0.60,0.02}{#1}}
\newcommand{\CommentTok}[1]{\textcolor[rgb]{0.56,0.35,0.01}{\textit{#1}}}
\newcommand{\CommentVarTok}[1]{\textcolor[rgb]{0.56,0.35,0.01}{\textbf{\textit{#1}}}}
\newcommand{\ConstantTok}[1]{\textcolor[rgb]{0.56,0.35,0.01}{#1}}
\newcommand{\ControlFlowTok}[1]{\textcolor[rgb]{0.13,0.29,0.53}{\textbf{#1}}}
\newcommand{\DataTypeTok}[1]{\textcolor[rgb]{0.13,0.29,0.53}{#1}}
\newcommand{\DecValTok}[1]{\textcolor[rgb]{0.00,0.00,0.81}{#1}}
\newcommand{\DocumentationTok}[1]{\textcolor[rgb]{0.56,0.35,0.01}{\textbf{\textit{#1}}}}
\newcommand{\ErrorTok}[1]{\textcolor[rgb]{0.64,0.00,0.00}{\textbf{#1}}}
\newcommand{\ExtensionTok}[1]{#1}
\newcommand{\FloatTok}[1]{\textcolor[rgb]{0.00,0.00,0.81}{#1}}
\newcommand{\FunctionTok}[1]{\textcolor[rgb]{0.13,0.29,0.53}{\textbf{#1}}}
\newcommand{\ImportTok}[1]{#1}
\newcommand{\InformationTok}[1]{\textcolor[rgb]{0.56,0.35,0.01}{\textbf{\textit{#1}}}}
\newcommand{\KeywordTok}[1]{\textcolor[rgb]{0.13,0.29,0.53}{\textbf{#1}}}
\newcommand{\NormalTok}[1]{#1}
\newcommand{\OperatorTok}[1]{\textcolor[rgb]{0.81,0.36,0.00}{\textbf{#1}}}
\newcommand{\OtherTok}[1]{\textcolor[rgb]{0.56,0.35,0.01}{#1}}
\newcommand{\PreprocessorTok}[1]{\textcolor[rgb]{0.56,0.35,0.01}{\textit{#1}}}
\newcommand{\RegionMarkerTok}[1]{#1}
\newcommand{\SpecialCharTok}[1]{\textcolor[rgb]{0.81,0.36,0.00}{\textbf{#1}}}
\newcommand{\SpecialStringTok}[1]{\textcolor[rgb]{0.31,0.60,0.02}{#1}}
\newcommand{\StringTok}[1]{\textcolor[rgb]{0.31,0.60,0.02}{#1}}
\newcommand{\VariableTok}[1]{\textcolor[rgb]{0.00,0.00,0.00}{#1}}
\newcommand{\VerbatimStringTok}[1]{\textcolor[rgb]{0.31,0.60,0.02}{#1}}
\newcommand{\WarningTok}[1]{\textcolor[rgb]{0.56,0.35,0.01}{\textbf{\textit{#1}}}}
\usepackage{graphicx}
\makeatletter
\def\maxwidth{\ifdim\Gin@nat@width>\linewidth\linewidth\else\Gin@nat@width\fi}
\def\maxheight{\ifdim\Gin@nat@height>\textheight\textheight\else\Gin@nat@height\fi}
\makeatother
% Scale images if necessary, so that they will not overflow the page
% margins by default, and it is still possible to overwrite the defaults
% using explicit options in \includegraphics[width, height, ...]{}
\setkeys{Gin}{width=\maxwidth,height=\maxheight,keepaspectratio}
% Set default figure placement to htbp
\makeatletter
\def\fps@figure{htbp}
\makeatother
\setlength{\emergencystretch}{3em} % prevent overfull lines
\providecommand{\tightlist}{%
  \setlength{\itemsep}{0pt}\setlength{\parskip}{0pt}}
\setcounter{secnumdepth}{-\maxdimen} % remove section numbering
\usepackage{amsmath,verbatim}

\usepackage{multirow}
\usepackage{fancyvrb}
\usepackage{manfnt}
\usepackage[normalem]{ulem}

\usepackage{hyperref}
\hypersetup{
	colorlinks=true,
	linkcolor=blue,
	urlcolor=blue}

%\usepackage[colorlinks=true]{hyperref}

\mode<presentation>{\usetheme{Malmoe}}

%\synctex=1

\setbeamertemplate{headline}{}


\setbeamerfont{footline}{size=\scriptsize}
\setbeamerfont{frametitle}{shape=\scshape}
\setbeamertemplate{itemize item}[circle]
\setbeamertemplate{itemize subitem}{\scriptsize$\diamond$}
\setbeamercovered{transparent}

\setbeamertemplate{navigation symbols}{}
\setbeamertemplate{footline}[frame number]{} 


\definecolor{forest}{rgb}{0, .5, 0}
\definecolor{brick}{rgb}{.5, 0, 0}
\definecolor{darkgreen}{rgb}{0, .5, 0}
\definecolor{darkred}{rgb}{.7, .15, .15}
\definecolor{darkblue}{rgb}{0, 0, .5}
\definecolor{Green}{rgb}{0.2,1,0.2}


\newcommand{\R}{\textsf{R}}
\newcommand{\RStudio}{\textsl{R Studio}}

\usepackage[english]{babel}
%\usepackage{palatino}
\usepackage[T1]{fontenc}


% make all tt fonts bold to look more like Verbatim
\usepackage{lmodern}
\renewcommand\ttfamily{\usefont{T1}{lmtt}{m}{n}}

% Comment these out if you don't want a slide with just the
% part/section/subsection/subsubsection title:
\AtBeginPart{
  \let\insertpartnumber\relax
  \let\partname\relax
  \frame{\partpage}
}
\AtBeginSection{
  \let\insertsectionnumber\relax
  \let\sectionname\relax
  \frame{\sectionpage}
}
\AtBeginSubsection{
  \let\insertsubsectionnumber\relax
  \let\subsectionname\relax
  \frame{\subsectionpage}
}

\newcommand{\columnsbegin}{\begin{columns}}
	\newcommand{\columnsend}{\end{columns}}

\newenvironment{twocol}[4]{
\begin{columns}[c]
\column{#1\textwidth}
#3
\column{#2\textwidth}
#4
\end{columns}
}

\def\begincols{\begin{columns}}
\def\begincol{\begin{column}}
\def\endcol{\end{column}}
\def\endcols{\end{columns}}
\ifLuaTeX
  \usepackage{selnolig}  % disable illegal ligatures
\fi
\IfFileExists{bookmark.sty}{\usepackage{bookmark}}{\usepackage{hyperref}}
\IfFileExists{xurl.sty}{\usepackage{xurl}}{} % add URL line breaks if available
\urlstyle{same}
\hypersetup{
  pdftitle={Unit 9: Inference for Categorical Data},
  pdfauthor={Statistics S-100 Teaching Team},
  hidelinks,
  pdfcreator={LaTeX via pandoc}}

\title{Unit 9: Inference for Categorical Data}
\author{Statistics S-100 Teaching Team}
\date{Summer 2024}

\begin{document}
\frame{\titlepage}

\begin{frame}[allowframebreaks]
  \tableofcontents[hideallsubsections]
\end{frame}
\hypertarget{introduction}{%
\section{Introduction}\label{introduction}}

\begin{frame}{Tools for assessing association}
\protect\hypertarget{tools-for-assessing-association}{}
\columnsbegin

\column{0.52\textwidth}

\footnotesize

We have covered methods for numerical outcomes:

\begin{itemize}
\item
  numerical outcome with a categorical predictor

  \begin{itemize}

    \footnotesize

    \item two-sample $t$-tests and ANOVA

    \item simple/multiple linear regression

    \end{itemize}
\item
  numerical outcome with a numerical predictor

  \begin{itemize}

    \footnotesize

    \item simple linear regression

    \end{itemize}
\item
  numerical outcome with several predictors, numerical or categorical

  \begin{itemize}

    \footnotesize

    \item multiple linear regression

    \end{itemize}
\end{itemize}

\column{0.52\textwidth}

\footnotesize

Next, methods for categorical outcomes:

\begin{itemize}
\item
  categorical outcome with a categorical predictor

  \begin{itemize}

    \footnotesize

    \item $\chi^2$ test of independence

    \item Fisher's exact test

    \item simple/multiple logistic regression

    \end{itemize}
\item
  binary outcome with a numerical predictor

  \begin{itemize}

    \footnotesize

    \item simple logistic regression

    \end{itemize}
\item
  binary outcome with several predictors, numerical or categorical

  \begin{itemize}

    \footnotesize

    \item multiple logistic regression

    \end{itemize}
\end{itemize}

\columnsend
\end{frame}

\hypertarget{inference-for-binomial-proportions}{%
\section{Inference for binomial
proportions}\label{inference-for-binomial-proportions}}

\begin{frame}{Fatal vehicle collisions}
\protect\hypertarget{fatal-vehicle-collisions}{}
\columnsbegin

\column{0.50\textwidth}

\footnotesize

According to the National Highway Traffic Safety Administration (NHTSA)
there were 33,949 fatal vehicle collisions across the US in 2018.

\vspace{0.25cm}

The cause of each accident is reported (e.g., distraction, drowsiness,
alcohol consumption, etc.) as well as the location.

\vspace{0.25cm}

In Massachusetts, 136 out of 341 fatal collisions involved alcohol
consumption.

\column{0.50\textwidth}

\footnotesize

Questions that can be addressed with inference\ldots{}

\begin{itemize}
\item
  What is the estimated \textbf{population proportion} of
  alcohol-related fatal collisions in MA?
\item
  What is the 95\% confidence interval for the estimated population
  proportion of alcohol-related fatal collisions in MA?
\item
  The nationwide proportion of alcohol-related fatal collisions is
  thought to be 0.33. Do the observed data for MA suggest that the
  probability of alcohol being involved in a fatal collision is greater
  in MA than nationwide?
\end{itemize}

\columnsend
\end{frame}

\begin{frame}{Inference for binomial proportions}
\protect\hypertarget{inference-for-binomial-proportions-1}{}
\small

The collision data are binomial data; ``success'' can be defined as
alcohol being involved.

Suppose \(X\) is a binomial random variable with parameters \(n\) and
\(p\), where \(n\) is the number of trials and \(p\) is the probability
of success.

\begin{itemize}
\item
  The parameter of interest is \(p\), the population probability of
  success; i.e., population probability of a fatal collision involving
  alcohol consumption.
\item
  The estimate of \(p\) from the observed sample is \(\hat{p} = x/n\),
  where \(x\) is the observed number of successes.
\end{itemize}

Inference for \(p\) can be done using the normal approximation to the
binomial, or directly using the binomial distribution.

\begin{itemize}
\item
  The normal approximation approach relies on the \textbf{Central Limit
  Theorem}.
\item
  The binomial approach is an example of an \textbf{exact} test, in
  which it is not necessary to approximate the sampling distribution of
  the test statistic.
\end{itemize}
\end{frame}

\begin{frame}{Normal theory approach (CLT for the sample proportion)}
\protect\hypertarget{normal-theory-approach-clt-for-the-sample-proportion}{}
\small

The sampling distribution of \(\hat{p}\) is approximately normal when

\begin{enumerate}
\item
  The sample observations are independent, and
\item
  At least 10 successes and 10 failures are expected in the sample:
  \(np \geq 10\) and
  \(n(1-p) \geq 10\).\footnote{This condition is commonly referred to as the success-failure condition.}
\end{enumerate}

Under these conditions, \(\hat{p}\) is approximately normally
distributed with mean \(p\) and standard deviation
\(\sqrt{\frac{p(1-p)}{n}}\).

Since \(p\) is unknown, it is necessary to substitute either \(\hat{p}\)
or \(p_0\) for \(p\) in the standard error term when computing
confidence intervals and test statistics.
\end{frame}

\begin{frame}[fragile]{Inference with the normal approximation}
\protect\hypertarget{inference-with-the-normal-approximation}{}
\columnsbegin

\column{0.50\textwidth}

\footnotesize

In the context of calculating CIs, substitute \(\hat{p}\) for \(p\).

\vspace{0.25cm}

An approximate two-sided 95\% confidence interval for \(p\) is given by
\[\hat{p} \pm 1.96 \sqrt{\frac{\hat{p}(1 - \hat{p})}{n}}. \]

\vspace{0.25cm}

With 95\% confidence, the interval (0.347, 0.453) captures the
population proportion of fatal collisions in MA that involved alcohol
consumption.

\column{0.50\textwidth}

\scriptsize

\begin{Shaded}
\begin{Highlighting}[]
\CommentTok{\#calculate confidence interval}
\FunctionTok{prop.test}\NormalTok{(}\AttributeTok{x =} \DecValTok{136}\NormalTok{, }\AttributeTok{n =} \DecValTok{341}\NormalTok{, }
          \AttributeTok{conf.level =} \FloatTok{0.95}\NormalTok{)}\SpecialCharTok{$}\NormalTok{conf.int}
\end{Highlighting}
\end{Shaded}

\begin{verbatim}
## [1] 0.3468427 0.4531286
## attr(,"conf.level")
## [1] 0.95
\end{verbatim}

\columnsend
\end{frame}

\begin{frame}[fragile]{Inference with the normal approximation\ldots{}}
\protect\hypertarget{inference-with-the-normal-approximation-1}{}
\columnsbegin

\column{0.45\textwidth}

\footnotesize

In the testing context, substitute \(p_0\) for \(p\).

\vspace{0.25cm}

The test statistic \(z\) for the null hypothesis \(H_0: p = p_0\) is
\[z = \dfrac{\hat{p} - p_0}{\sqrt{\dfrac{(p_0)(1 - p_0)}{n}}} \]

\begin{itemize}
\item
  If the proportion of fatal collisions in MA that involved alcohol
  consumption were actually 0.33, there would be only a 0.0041
  probability of observing a sample proportion of alcohol-related fatal
  collisions equal to 0.399 or larger.
\item
  Thus, these data suggest that the proportion of alcohol-related fatal
  collisions in MA is higher than the nationwide proportion of 0.33.
\end{itemize}

\column{0.55\textwidth}

\scriptsize

\begin{Shaded}
\begin{Highlighting}[]
\CommentTok{\#conduct hypothesis test}
\FunctionTok{prop.test}\NormalTok{(}\AttributeTok{x =} \DecValTok{136}\NormalTok{, }\AttributeTok{n =} \DecValTok{341}\NormalTok{, }\AttributeTok{p =} \FloatTok{0.33}\NormalTok{, }
          \AttributeTok{alternative =} \StringTok{"greater"}\NormalTok{)}
\end{Highlighting}
\end{Shaded}

\begin{verbatim}
## 
##  1-sample proportions test with continuity correction
## 
## data:  136 out of 341, null probability 0.33
## X-squared = 6.9981, df = 1, p-value = 0.00408
## alternative hypothesis: true p is greater than 0.33
## 95 percent confidence interval:
##  0.3547446 1.0000000
## sample estimates:
##        p 
## 0.398827
\end{verbatim}

\columnsend
\end{frame}

\begin{frame}[fragile]{Exact inference for binomial data}
\protect\hypertarget{exact-inference-for-binomial-data}{}
\footnotesize

Definition of the \(p\)-value: the probability of observing 136 or more
successes out of 341 trials if the null hypothesis \(H_0: p = 0.33\)
were true.

\vspace{0.1cm}

\scriptsize

\begin{Shaded}
\begin{Highlighting}[]
\CommentTok{\#use pbinom( )}
\FunctionTok{pbinom}\NormalTok{(}\DecValTok{135}\NormalTok{, }\DecValTok{341}\NormalTok{, }\AttributeTok{p =} \FloatTok{0.33}\NormalTok{, }\AttributeTok{lower.tail =} \ConstantTok{FALSE}\NormalTok{)}
\end{Highlighting}
\end{Shaded}

\begin{verbatim}
## [1] 0.004507281
\end{verbatim}

\begin{Shaded}
\begin{Highlighting}[]
\CommentTok{\#use binom.test( )}
\FunctionTok{binom.test}\NormalTok{(}\AttributeTok{x =} \DecValTok{136}\NormalTok{, }\AttributeTok{n =} \DecValTok{341}\NormalTok{, }\AttributeTok{p =} \FloatTok{0.33}\NormalTok{, }\AttributeTok{alternative =} \StringTok{"greater"}\NormalTok{)}
\end{Highlighting}
\end{Shaded}

\begin{verbatim}
## 
##  Exact binomial test
## 
## data:  136 and 341
## number of successes = 136, number of trials = 341, p-value = 0.004507
## alternative hypothesis: true probability of success is greater than 0.33
## 95 percent confidence interval:
##  0.3545283 1.0000000
## sample estimates:
## probability of success 
##               0.398827
\end{verbatim}
\end{frame}

\begin{frame}{Inference for the difference of two proportions}
\protect\hypertarget{inference-for-the-difference-of-two-proportions}{}
\small

The one-sample \(z\)-test for a population proportion is analogous to
the one-sample \(t\)-test for a population mean:

\begin{itemize}
\tightlist
\item
  Sample statistic: \(\hat{p}\), Parameter: \(p\), Null hypothesis:
  \(H_0: p = p_0\)
\item
  Sample statistic: \(\overline{x}\), Parameter: \(\mu\), Null
  hypothesis: \(H_0: \mu = \mu_0\)
\end{itemize}

Similarly, there exists a two-sample \(z\)-test for the difference of
population proportions that is analogous to the two-sample \(t\)-test
for the difference of population means:

\begin{itemize}
\tightlist
\item
  Sample statistic: \(\hat{p}_1 - \hat{p}_2\), Parameter: \(p_1 - p_2\),
  Null hypothesis: \(H_0: p_1 - p_2 = 0\)
\item
  Sample statistic: \(\overline{x}_1 - \overline{x}_2\), Parameter:
  \(\mu_1 - \mu_2\), Null hypothesis: \(H_0: \mu_1 - \mu_2 = 0\)
\end{itemize}

For completeness, slides 13-14 show the details of the two-sample
proportions test. We will focus on learning a more flexible approach for
analyzing the association between two categorical variables.
\end{frame}

\begin{frame}{Inference for the difference of two proportions\ldots{}}
\protect\hypertarget{inference-for-the-difference-of-two-proportions-1}{}
\small

The normal model can be applied to \(\hat{p}_1 - \hat{p}_2\) if

\begin{enumerate}
\item
  The two samples are independent, the observations in each sample are
  independent, and
\item
  At least 10 successes and 10 failures are expected in each sample.
\end{enumerate}

The standard error of the difference in sample proportions is
\[\sqrt{\dfrac{p_1(1 - p_1)}{n_1} + \dfrac{p_2(1 - p_2)}{n_2}} \]

In hypothesis testing, the following estimate of \(p\) is used to
compute the standard error:
\[\hat{p} = \dfrac{n_1\hat{p}_1 + n_2\hat{p}_2}{n_1 + n_2} = \dfrac{x_1 + x_2}{n_1 + n_2} \]
\end{frame}

\begin{frame}[fragile]{Fatal vehicle collisions\ldots{}}
\protect\hypertarget{fatal-vehicle-collisions-1}{}
\small

Does the population proportion of alcohol-related fatal collisions
differ between MA and UT?

\scriptsize

\begin{verbatim}
##      Cause
## State Alcohol Not Alcohol Sum
##   MA      136         205 341
##   UT       58         179 237
##   Sum     194         384 578
\end{verbatim}

\begin{Shaded}
\begin{Highlighting}[]
\CommentTok{\#analyze the data}
\FunctionTok{prop.test}\NormalTok{(}\AttributeTok{x =} \FunctionTok{c}\NormalTok{(}\DecValTok{136}\NormalTok{, }\DecValTok{58}\NormalTok{), }\AttributeTok{n =} \FunctionTok{c}\NormalTok{(}\DecValTok{341}\NormalTok{, }\DecValTok{237}\NormalTok{))}
\end{Highlighting}
\end{Shaded}

\begin{verbatim}
## 
##  2-sample test for equality of proportions with continuity correction
## 
## data:  c(136, 58) out of c(341, 237)
## X-squared = 14.207, df = 1, p-value = 0.0001637
## alternative hypothesis: two.sided
## 95 percent confidence interval:
##  0.07504718 0.23315531
## sample estimates:
##    prop 1    prop 2 
## 0.3988270 0.2447257
\end{verbatim}
\end{frame}

\hypertarget{inference-for-two-way-tables}{%
\section{Inference for two-way
tables}\label{inference-for-two-way-tables}}

\begin{frame}{Inference for two-way tables}
\protect\hypertarget{inference-for-two-way-tables-1}{}
\small

A two-way table summarizes information about the relationship between
two categorical variables.

Testing for a difference between \(p_1\) and \(p_2\) is equivalent to
testing for association in a two-way table that has two rows and two
columns.

\vspace{0.5cm}

\begin{center}
\begin{tabular}{l|cc|c} 
   & \textbf{Outcome: Success} & \textbf{Outcome: Failure} & \textbf{Total}\\ \hline
  \textbf{Group 1} & $x_1$ & $n_1 - x_1$ & $n_1$  \\
  \textbf{Group 2} & $x_2$ &  $n_2 - x_2$ & $n_2$ \\ \hline
  \textbf{Total} & $x_1 + x_2$ & $(n_1 - x_1) + (n_2 - x_2)$ & $n_1 + n_2$  \\ 
\end{tabular}\\
\end{center}
\end{frame}

\begin{frame}{Treating HIV\(^{+}\) infants}
\protect\hypertarget{treating-hiv-infants}{}
\columnsbegin

\column{0.50\textwidth}

\footnotesize

In resource-limited settings, single-dose nevirapine is given to an
HIV\(^{+}\) woman during birth to prevent mother-to-child transmission
of the virus.

\begin{itemize}
\item
  Exposure of the infant to nevirapine (NVP) may foster the growth of
  resistant strains of the virus in the child.
\item
  If the child is HIV\(^+\), should they be treated with nevirapine or a
  more expensive drug, lopinarvir (LPV)?
\end{itemize}

In this setting, the possible outcomes are virologic failure (the virus
becomes resistant) versus stable disease (virus growth is prevented).

\column{0.52\textwidth}

\footnotesize

The following table summarizes the results of a 2012 study comparing NVP
versus LPV in treatment of HIV-infected
infants.\footnote{Violari, et al. \textit{NEJM} 2012; 366: 2380-2389.}
Children were randomized to receive either NVP or LPV.

\begin{center}
\begin{tabular}{l|cc|c} 
   & \textbf{Stable Disease} & \textbf{Virologic Failure} & \textbf{Total}\\ \hline
  \textbf{NVP} & 87 & 60 & 147  \\
  \textbf{LPV} & 113 & 27 & 140 \\ \hline
  \textbf{Total} & 200 & 87 & 287  \\ 
\end{tabular}\\
\end{center}

\columnsend
\end{frame}

\begin{frame}{Formulating hypotheses in a two-way table}
\protect\hypertarget{formulating-hypotheses-in-a-two-way-table}{}
\small

The main question of interest:

\begin{itemize}
\tightlist
\item
  Do the data support the claim of a difference in outcome by treatment?
\end{itemize}

If there is no difference in outcome by treatment, then knowing
treatment provides no information about outcome; treatment assignment
and outcome are \emph{independent} (i.e., \emph{not associated}).

\begin{itemize}
\item
  \(H_{0}\): Treatment and outcome are not associated.
\item
  \(H_{A}\): Treatment and outcome are associated.

  \begin{itemize}
  \tightlist
  \item
    This is inherently a two-sided alternative.
  \end{itemize}
\end{itemize}
\end{frame}

\begin{frame}{The \(\chi^2\) test of independence}
\protect\hypertarget{the-chi2-test-of-independence}{}
\small

In the \(\chi^2\) test, the observed number of cell counts are compared
to the number of \textbf{expected} cell counts, where the expected
counts are calculated under the null hypothesis.

\begin{itemize}
\item
  The test statistic quantifies how far the observed results deviate
  from what is expected under the null hypothesis.
\item
  A larger test statistic represents stronger evidence against the null
  hypothesis of independence.
\end{itemize}
\end{frame}

\begin{frame}{Expected cell counts}
\protect\hypertarget{expected-cell-counts}{}
\small

If treatment had no effect on outcome, what would we expect to see?

\begin{itemize}
\item
  Let \(A\) = \{assignment to NVP\}
\item
  Let \(B\) = \{virologic failure\}
\end{itemize}

Under the hypothesis of independence,

\[ P(A\text{ and } B) = P(A) \times P(B) = \left(\frac{147}{287}\right) \left(\frac{87}{287}\right)\]

The expected cell count in the upper right corner would be

\[(287) \left(\frac{147}{287}\right) \left(\frac{87}{287}\right) = 44.56\]

What about the other cells?
\end{frame}

\begin{frame}{Formula for expected cell counts}
\protect\hypertarget{formula-for-expected-cell-counts}{}
\columnsbegin

\column{0.45\textwidth}

\footnotesize

The expected count for the \(i^{th}\) row and \(j^{th}\) column is

\[E_{i, j} = \dfrac{(\text{row $i$ total}) \times (\text{column $j$ total}) }{n}, \]
where \(n\) is the total number of observations.

\column{0.55\textwidth}

\footnotesize

\begin{center}
\begin{tabular}{l|cc|c} 
   & \textbf{Stable Disease} & \textbf{Virologic Failure} & \textbf{Total}\\ \hline
  \textbf{NVP} & 87 \textcolor{blue}{(102.44)} & 60 \textcolor{blue}{(44.56)} & 147  \\
  \textbf{LPV} & 113 \textcolor{blue}{(97.56)} & 27 \textcolor{blue}{(42.44)} & 140 \\ \hline
  \textbf{Total} & 200 & 87 & 287  \\ 
\end{tabular}\\
\end{center}

\columnsend
\end{frame}

\begin{frame}{Visual comparison of observed versus expected}
\protect\hypertarget{visual-comparison-of-observed-versus-expected}{}
\includegraphics{unit_09_inference_cat_files/figure-beamer/unnamed-chunk-6-1.pdf}
\end{frame}

\begin{frame}{The \(\chi^2\) test statistic}
\protect\hypertarget{the-chi2-test-statistic}{}
\columnsbegin

\column{0.50\textwidth}

\footnotesize

The \textbf{\(\chi^2\) test statistic} is calculated as
\[\chi^2 = \sum_{i = 1}^r \sum_{j = 1}^c \dfrac{(O_{i, j} - E_{i, j})^2}{E_{i, j}}, \]
and is approximately distributed \(\chi^2\) with degrees of freedom
\((r - 1)(c - 1)\), where \(r\) is the number of rows and \(c\) is the
number of columns.

\begin{itemize}
\item
  \(O_{i, j}\) represents the observed count in row \(i\), column \(j\).
\item
  \(E_{i, j}\) represents the expected count in row \(i\), column \(j\).
\end{itemize}

\column{0.50\textwidth}

\footnotesize

Assumptions for the \(\chi^2\) test:

\begin{itemize}
\item
  \emph{Independence}. Each case that contributes a count to the table
  must be independent of all other cases in the table.
\item
  \emph{Sample size}. Each expected cell count must be greater than or
  equal to
  10.\footnote{Some sources use a less strict sample size condition. For example, the \texttt{chisq.test()} function only shows a warning if one of the expected counts is smaller than 5.}

  \begin{itemize}

    \footnotesize

    \item For tables larger than $2 \times 2$, it is appropriate to use the test if no more than 1/5 of the expected counts are less than 5, and all expected counts are greater than 1.

    \end{itemize}
\end{itemize}

These assumptions must be met for the test statistic to be approximately
distributed \(\chi^2\).

\columnsend
\end{frame}

\begin{frame}[fragile]{The \(\chi^2\) test in \textsf{R}}
\protect\hypertarget{the-chi2-test-in}{}
\scriptsize

\begin{Shaded}
\begin{Highlighting}[]
\NormalTok{hiv.table }\OtherTok{\textless{}{-}} \FunctionTok{matrix}\NormalTok{(}\FunctionTok{c}\NormalTok{(}\DecValTok{87}\NormalTok{, }\DecValTok{113}\NormalTok{, }\DecValTok{60}\NormalTok{, }\DecValTok{27}\NormalTok{), }\AttributeTok{nrow =} \DecValTok{2}\NormalTok{, }\AttributeTok{ncol =} \DecValTok{2}\NormalTok{, }\AttributeTok{byrow =}\NormalTok{ F)}
\FunctionTok{dimnames}\NormalTok{(hiv.table) }\OtherTok{\textless{}{-}} \FunctionTok{list}\NormalTok{(}\StringTok{"Drug"} \OtherTok{=} \FunctionTok{c}\NormalTok{(}\StringTok{"NVP"}\NormalTok{, }\StringTok{"LPV"}\NormalTok{),}
                           \StringTok{"Outcome"} \OtherTok{=} \FunctionTok{c}\NormalTok{(}\StringTok{"Stable Disease"}\NormalTok{, }\StringTok{"V. Failure"}\NormalTok{))}
\FunctionTok{chisq.test}\NormalTok{(hiv.table)}
\end{Highlighting}
\end{Shaded}

\begin{verbatim}
## 
##  Pearson's Chi-squared test with Yates' continuity correction
## 
## data:  hiv.table
## X-squared = 14.733, df = 1, p-value = 0.0001238
\end{verbatim}

\begin{Shaded}
\begin{Highlighting}[]
\FunctionTok{chisq.test}\NormalTok{(hiv.table)}\SpecialCharTok{$}\NormalTok{expected}
\end{Highlighting}
\end{Shaded}

\begin{verbatim}
##      Outcome
## Drug  Stable Disease V. Failure
##   NVP      102.43902   44.56098
##   LPV       97.56098   42.43902
\end{verbatim}
\end{frame}

\begin{frame}{Residuals in the \(\chi^2\) test}
\protect\hypertarget{residuals-in-the-chi2-test}{}
\small

For each cell in a table, the \textbf{residual} equals
\[\dfrac{O_{i, j} - E_{i, j}}{\sqrt{E_{i,j}}}. \]

Residuals with a large magnitude contribute the most to the \(\chi^2\)
statistic.

\begin{itemize}
\item
  If a residual is positive, the observed value is greater than the
  expected value.
\item
  If a residual is negative, the observed value is less than the
  expected.
\end{itemize}
\end{frame}

\begin{frame}[fragile]{Residuals in the \(\chi^2\) test\ldots{}}
\protect\hypertarget{residuals-in-the-chi2-test-1}{}
\footnotesize

\begin{Shaded}
\begin{Highlighting}[]
\FunctionTok{chisq.test}\NormalTok{(hiv.table)}\SpecialCharTok{$}\NormalTok{residuals}
\end{Highlighting}
\end{Shaded}

\begin{verbatim}
##      Outcome
## Drug  Stable Disease V. Failure
##   NVP      -1.525412   2.312824
##   LPV       1.563082  -2.369939
\end{verbatim}

\small

Examining the residuals can be informative for understanding direction
of association.

\begin{itemize}
\tightlist
\item
  Which drug is associated with stable disease; i.e., which drug should
  be recommended for treatment of HIV-infected infants?
\end{itemize}
\end{frame}

\begin{frame}{Treating \emph{C. difficile} infection}
\protect\hypertarget{treating-c.-difficile-infection}{}
\small

\emph{Clostridium difficile} is a bacterium that causes inflammation of
the colon. Antibiotic treatment is typically not effective. Infusion of
feces from healthy donors has been reported as an effective treatment.

A randomized trial was conducted to compare the efficacy of donor-feces
infusion versus vancomycin, the antibiotic typically prescribed to treat
\emph{C. difficile} infection.

\begin{table}[h]
    \centering
    \begin{tabular}{rrrr}
        \hline
        & Cured & Uncured & Sum \\ 
        \hline
        Fecal Infusion & 13 & 3 & 16 \\ 
        Vancomycin & 4 & 9 & 13 \\ 
        Sum & 17 & 12 & 29 \\ 
        \hline
    \end{tabular}
    \caption{Fecal Infusion Study Results} 
    \label{fecalStudyResultsTest}
\end{table}

Can a \(\chi^2\) test be used to analyze these results?
\end{frame}

\begin{frame}{Fisher's exact test}
\protect\hypertarget{fishers-exact-test}{}
\small

Fisher's exact test works even when sample sizes are
small.\footnote{In this course, Fisher's exact test is only discussed in the context of $2 \times 2$ tables.}

In this particular experiment, we observed 17 cured individuals (out of
29 total) when 16 were assigned to fecal infusion and 13 to vancomycin.

\begin{itemize}
\item
  Under \(H_0: p_1 = p_2\), individuals in one treatment group are just
  as likely to be cured as individuals in the other group.
\item
  If \(H_0\) is true (and the study had the same setup):

  \begin{itemize}

    \footnotesize

    \item What is the probability that of the 17 cured individuals, 13 were in the fecal infusion group?

    \item What are the possible sets of results that indicate stronger evidence in favor of fecal infusion?

    \item What is the probability of seeing even stronger evidence in favor of fecal infusion as an effective treatment?

    \end{itemize}
\end{itemize}

The \(p\)-value for Fisher's exact test is calculated by adding together
the individual conditional probabilities of obtaining each table that is
\textbf{as extreme or more extreme than the one observed}, under the
null hypothesis and given that the marginal totals are considered fixed.
\end{frame}

\begin{frame}{The hypergeometric distribution}
\protect\hypertarget{the-hypergeometric-distribution}{}
\small

Let \(X\) represent the number of successes in a series of repeated
Bernoulli trials, where sampling is done without replacement.

\begin{itemize}
\item
  In a population of size \(N\), there are \(m\) total successes.
\item
  What is the probability of observing exactly \(k\) successes when
  drawing a sample of size \(n\)?
\end{itemize}

For example, imagine an urn with \(m\) white balls and \(N - m\) red
balls. Draw \(n\) balls without replacement. What is the probability of
observing \(k\) white balls in the sample?

\begin{table}[h!]
\begin{center}
\begin{tabular}{l|cc|c} 
   & \textbf{White Ball} & \textbf{Red Ball} & \textbf{Total}\\ \hline
  \textbf{Sampled} & $k$ & \textcolor{gray}{$n - k$}  & $n$  \\
  \textbf{Not Sampled} & \textcolor{gray}{$m - k$} & \textcolor{gray}{$N - n - (m - k)$} & \textcolor{gray}{$N - n$} \\ \hline
  \textbf{Total} & $m$ & $N - m$ & $N$  \\ 
\end{tabular}\\
\end{center}
\end{table}
\end{frame}

\begin{frame}[fragile]{The hypergeometric distribution\ldots{}}
\protect\hypertarget{the-hypergeometric-distribution-1}{}
\small

To calculate \(P(X = k)\) where \(X \sim \text{HGeom}(m, N - m, n)\),
use \texttt{dhyper( )}:

\vspace{0.2cm}

\scriptsize

\begin{Shaded}
\begin{Highlighting}[]
\FunctionTok{dhyper}\NormalTok{(k, m, N }\SpecialCharTok{{-}}\NormalTok{ m, n)}
\end{Highlighting}
\end{Shaded}

\small

Suppose the urn contains 10 white balls, 15 red balls, and a sample of
size 8 is drawn. What is the probability of observing 5 white balls in
the sample?

\scriptsize

\begin{Shaded}
\begin{Highlighting}[]
\FunctionTok{dhyper}\NormalTok{(}\DecValTok{5}\NormalTok{, }\DecValTok{10}\NormalTok{, }\DecValTok{25} \SpecialCharTok{{-}} \DecValTok{10}\NormalTok{, }\DecValTok{8}\NormalTok{)}
\end{Highlighting}
\end{Shaded}

\begin{verbatim}
## [1] 0.1060121
\end{verbatim}
\end{frame}

\begin{frame}[fragile]{Treating \emph{C. difficile} infection\ldots{}}
\protect\hypertarget{treating-c.-difficile-infection-1}{}
\small

Given that 17 individuals out of 29 were cured and that 16 individuals
were in the fecal infusion group (and that \(H_0\) is true), what is the
probability that 13 of the cured individuals were in the fecal infusion
group?

\begin{itemize}
\item
  \(N = 29\), \(m = 17\), and \(n = 16\)
\item
  Calculate \(P(X = 13)\).
\end{itemize}

\scriptsize

\begin{Shaded}
\begin{Highlighting}[]
\CommentTok{\#probability of observed results}
\FunctionTok{dhyper}\NormalTok{(}\DecValTok{13}\NormalTok{, }\DecValTok{17}\NormalTok{, }\DecValTok{29} \SpecialCharTok{{-}} \DecValTok{17}\NormalTok{, }\DecValTok{16}\NormalTok{)}
\end{Highlighting}
\end{Shaded}

\begin{verbatim}
## [1] 0.007715441
\end{verbatim}
\end{frame}

\begin{frame}[fragile]{Fisher's exact test\ldots{}}
\protect\hypertarget{fishers-exact-test-1}{}
\small

For a one-sided \(p\)-value\ldots{}

\begin{itemize}
\tightlist
\item
  Sum the probabilities of the results as or more extreme than those
  observed; that is, the probability of the observed table and that of
  all tables that are more extreme in the direction specified by the
  alternative hypothesis.
\end{itemize}

\scriptsize

\begin{Shaded}
\begin{Highlighting}[]
\CommentTok{\#one{-}sided p{-}value}
\FunctionTok{phyper}\NormalTok{(}\DecValTok{12}\NormalTok{, }\DecValTok{17}\NormalTok{, }\DecValTok{29} \SpecialCharTok{{-}} \DecValTok{17}\NormalTok{, }\DecValTok{16}\NormalTok{, }\AttributeTok{lower.tail =} \ConstantTok{FALSE}\NormalTok{)}
\end{Highlighting}
\end{Shaded}

\begin{verbatim}
## [1] 0.008401063
\end{verbatim}

\small

For a two-sided \(p\)-value\ldots{}

\begin{itemize}
\tightlist
\item
  Consider extreme tables to be all tables with probabilities less than
  that of the observed; sum the probabilities of tables representing
  results as or more extreme than those observed.
\end{itemize}
\end{frame}

\begin{frame}[fragile]{Treating \emph{C. difficile} infection\ldots{}}
\protect\hypertarget{treating-c.-difficile-infection-2}{}
\scriptsize

\begin{Shaded}
\begin{Highlighting}[]
\CommentTok{\#enter the data}
\NormalTok{infusion.table }\OtherTok{=} \FunctionTok{matrix}\NormalTok{(}\FunctionTok{c}\NormalTok{(}\DecValTok{13}\NormalTok{, }\DecValTok{3}\NormalTok{, }\DecValTok{4}\NormalTok{, }\DecValTok{9}\NormalTok{), }\AttributeTok{nrow =} \DecValTok{2}\NormalTok{, }\AttributeTok{ncol =} \DecValTok{2}\NormalTok{, }\AttributeTok{byrow =}\NormalTok{ T)}
\FunctionTok{dimnames}\NormalTok{(infusion.table) }\OtherTok{=} \FunctionTok{list}\NormalTok{(}\StringTok{"Outcome"} \OtherTok{=} \FunctionTok{c}\NormalTok{(}\StringTok{"Cured"}\NormalTok{, }\StringTok{"Uncured"}\NormalTok{),}
                                \StringTok{"Treatment"} \OtherTok{=} \FunctionTok{c}\NormalTok{(}\StringTok{"Fecal Infusion"}\NormalTok{, }
                                                \StringTok{"Vancomycin"}\NormalTok{))}

\FunctionTok{fisher.test}\NormalTok{(infusion.table, }\AttributeTok{alternative =} \StringTok{"greater"}\NormalTok{)}
\end{Highlighting}
\end{Shaded}

\begin{verbatim}
## 
##  Fisher's Exact Test for Count Data
## 
## data:  infusion.table
## p-value = 0.008401
## alternative hypothesis: true odds ratio is greater than 1
## 95 percent confidence interval:
##  1.735233      Inf
## sample estimates:
## odds ratio 
##   8.848725
\end{verbatim}
\end{frame}

\begin{frame}[fragile]{Treating \emph{C. difficile} infection\ldots{}}
\protect\hypertarget{treating-c.-difficile-infection-3}{}
\columnsbegin

\column{0.30\textwidth}

\scriptsize

\begin{verbatim}
##     k     prob
## 1   0 0.000000
## 2   1 0.000000
## 3   2 0.000000
## 4   3 0.000000
## 5   4 0.000035
## 6   5 0.001094
## 7   6 0.012036
## 8   7 0.063046
## 9   8 0.177317
## 10  9 0.283708
## 11 10 0.264794
## 12 11 0.144433
## 13 12 0.045135
## 14 13 0.007715
## 15 14 0.000661
## 16 15 0.000024
## 17 16 0.000000
\end{verbatim}

\column{0.70\textwidth}

\scriptsize

\begin{Shaded}
\begin{Highlighting}[]
\CommentTok{\#P(X leq 5) + P(X geq 13)}
\FunctionTok{phyper}\NormalTok{(}\DecValTok{5}\NormalTok{, }\DecValTok{17}\NormalTok{, }\DecValTok{29} \SpecialCharTok{{-}} \DecValTok{17}\NormalTok{, }\DecValTok{16}\NormalTok{) }\SpecialCharTok{+} 
  \FunctionTok{phyper}\NormalTok{(}\DecValTok{12}\NormalTok{, }\DecValTok{17}\NormalTok{, }\DecValTok{29} \SpecialCharTok{{-}} \DecValTok{17}\NormalTok{, }\DecValTok{16}\NormalTok{, }\AttributeTok{lower.tail =}\NormalTok{ F)}
\end{Highlighting}
\end{Shaded}

\begin{verbatim}
## [1] 0.009530323
\end{verbatim}

\begin{Shaded}
\begin{Highlighting}[]
\CommentTok{\#two{-}sided p{-}value}
\FunctionTok{fisher.test}\NormalTok{(infusion.table)}
\end{Highlighting}
\end{Shaded}

\begin{verbatim}
## 
##  Fisher's Exact Test for Count Data
## 
## data:  infusion.table
## p-value = 0.00953
## alternative hypothesis: true odds ratio is not equal to 1
## 95 percent confidence interval:
##   1.373866 78.811505
## sample estimates:
## odds ratio 
##   8.848725
\end{verbatim}

\columnsend
\end{frame}

\hypertarget{measures-of-effect-size-in-two-by-two-tables}{%
\section{Measures of effect size in two-by-two
tables}\label{measures-of-effect-size-in-two-by-two-tables}}

\begin{frame}{Measures of effect size for categorical outcomes}
\protect\hypertarget{measures-of-effect-size-for-categorical-outcomes}{}
\small

Recent article:
\textcolor{blue}{\href{https://www.nytimes.com/2023/07/13/health/aspartame-cancer-who-sweetener.html}{"Aspartame Is a Possible Cause of Cancer in Humans, a W.H.O. Agency Says"}}

\footnotesize

\begin{itemize}
\item
  Many studies have investigated potential links between artificial
  sweeteners and cancer.
\item
  ``The highest category of aspartame intake (\(\geq 143\) mg/day) was
  associated with elevated relative risk of non-Hodgkin lymphoma (RR =
  1.64, 95\% CI 1.17 - 2.29) in
  men.''\footnote{\href{https://health.gov/our-work/nutrition-physical-activity/dietary-guidelines/previous-dietary-guidelines/2015/advisory-report/appendix-e-2/appendix-e-241}{Study referenced in 2015 dietary guidelines advisory report}}
\item
  ``High consumption of aspartame was associated with stomach cancer (OR
  = 2.27, 95\% CI 0.99 - 5.44), while a lower risk was observed for
  breast cancer (OR = 0.28, 95\% CI 0.08 -
  0.83).''\footnote{\href{https://onlinelibrary.wiley.com/doi/10.1002/ijc.34577}{Study by Palomar-Cros, et al.}}
\end{itemize}

\small

Results from studies done to investigate the effect of a risk factor on
an outcome of interest are often reported as relative risks (RRs) or
odds ratios (ORs).

\begin{itemize}
\tightlist
\item
  Important caveat: RR and OR should always be examined in the context
  of the \textbf{absolute risk}; i.e., estimate of risk in the baseline
  group.
\end{itemize}
\end{frame}

\begin{frame}{Relative risk in a \(2 \times 2\) table}
\protect\hypertarget{relative-risk-in-a-2-times-2-table}{}
\small

The \textbf{relative risk (RR)} is a measure of the risk of a certain
event occurring in one group relative to the risk of the event occuring
in another group.

The risk of virologic failure among the NVP group is
\[\dfrac{\text{\# in NVP group and had virologic failure}}{\text{total \# in NVP group}} = \dfrac{60}{147} = 0.408 \]

The risk of virologic failure among the LPV group is
\[\dfrac{\text{\# in LPV group and had virologic failure}}{\text{total \# in LPV group}} = \dfrac{27}{140} = 0.193\]

Thus, the relative risk of virologic failure comparing NVP to LPV is
\(0.408/0.193 = 2.11\).

\begin{itemize}
\tightlist
\item
  Children treated with NVP are estimated to be more than twice as
  likely to experience virologic failure.
\end{itemize}
\end{frame}

\begin{frame}{Confidence interval for relative risk}
\protect\hypertarget{confidence-interval-for-relative-risk}{}
\small

Let \(y_1\) and \(y_2\) represent the observed number of successes in
two groups of size \(n_1\) and \(n_2\). Let the risk (of the event
defined as success) in each group be represented as
\(\hat{p}_1 = y_1/n_1\) and \(\hat{p}_2 = y_2/n_2\) and the estimated
relative risk be \(\widehat{RR} = \hat{p}_1/\hat{p}_2\).

\[\text{SE}_{\text{log($\widehat{RR}$)}} = \sqrt{\dfrac{1 - \hat{p}_1}{y_1} + \dfrac{1 - \hat{p}_2}{y_2}} \]

A \(100(1-\alpha)\)\% confidence interval for
log(RR)\footnote{This CI is valid when all expected cell counts $\geq 10$.}
is given by
\[\log(\widehat{RR}) \pm \left( z^\star \times  \text{SE}_{\text{log($\widehat{RR}$)}} \right) \]

To obtain the confidence interval for RR, exponentiate the bounds of the
CI for log(RR).
\end{frame}

\begin{frame}[fragile]{Confidence interval for relative risk\ldots{}}
\protect\hypertarget{confidence-interval-for-relative-risk-1}{}
\small

Compute a 95\% CI for the relative risk of virologic failure comparing
NPV to LPV.

\[\text{SE}_{\text{log($\widehat{RR}$)}} = \sqrt{\dfrac{1 - \hat{p}_1}{y_1} + \dfrac{1 - \hat{p}_2}{y_2}} = \sqrt{\dfrac{1 - 0.408}{60} + \dfrac{1 - 0.193}{27}} = 0.199\]

95\% CI for log(RR):
\[\log(2.11) \pm (1.96)(0.199) \rightarrow (0.358, 1.139)\]

95\% CI for RR: \[(e^{0.358}, e^{1.139}) \rightarrow (1.430, 3.125) \]

\scriptsize

\begin{Shaded}
\begin{Highlighting}[]
\FunctionTok{library}\NormalTok{(epitools)}
\FunctionTok{riskratio}\NormalTok{(hiv.table, }\AttributeTok{rev =} \StringTok{"rows"}\NormalTok{)}\SpecialCharTok{$}\NormalTok{measure}
\end{Highlighting}
\end{Shaded}

\begin{verbatim}
##      risk ratio with 95% C.I.
## Drug  estimate   lower    upper
##   LPV 1.000000      NA       NA
##   NVP 2.116402 1.43177 3.128405
\end{verbatim}
\end{frame}

\begin{frame}{Odds and the odds ratio in a \(2 \times 2\) table}
\protect\hypertarget{odds-and-the-odds-ratio-in-a-2-times-2-table}{}
\small

The \textbf{odds} of an event \(E\) are \(\frac{P(E)}{1 - P(E)}\).

The \textbf{odds ratio (OR)} is a measure of the odds of a certain event
occurring in one group relative to the odds of the event occurring in
another group.

The odds of virologic failure among the NVP group is
\[\dfrac{\text{\# in NVP group and had virologic failure}}{\text{\# in NVP group and did not have virologic failure}} = \dfrac{60}{87} = 0.690 \]

The odds of virologic failure among the LPV group is
\[\dfrac{\text{\# in LPV group and had virologic failure}}{\text{\# in LPV group and did not have virologic failure}} = \dfrac{27}{113} = 0.239\]

Thus, the odds ratio of virologic failure comparing NVP to LPV is
\(0.690/0.239 = 2.89\).

\begin{itemize}
\tightlist
\item
  The odds of virologic failure when treated with NVP are almost three
  times as large as the odds of virologic failure when treated with LPV.
\end{itemize}
\end{frame}

\begin{frame}{Odds and probabilities}
\protect\hypertarget{odds-and-probabilities}{}
\small

With some algebra, it is possible to show the following relationship:
\[\text{odds} = \dfrac{p}{1-p} \qquad p = \frac{\text{odds}}{1 + \text{odds}}\]

Probabilities and odds increase or decrease together.

\begin{itemize}
\tightlist
\item
  Note that while probabilities always have values between 0 and 1
  (inclusive), odds can be much larger than 1.
\end{itemize}

\footnotesize

\begin{table}
\centering
\begin{tabular}{c|c|c}
\textbf{Probability} & \textbf{Odds} = $p/(1-p)$ &\textbf{Odds}\\
\hline
0 &0/1 = 0 &0 \\
1/100 = 0.01 &1/99 = 0.0101 &1 : 99 \\
1/10 = 0.10 & 1/9 = 0.11 & 1 : 9 \\
1/4 &1/3 &1 : 3 \\
1/3 &1/2 &1 : 2 \\
1/2 &$(\frac{1}{2})/(\frac{1}{2})=1$ &1 : 1 \\
2/3 &$(2/3)/(1/3)=2$ &2 : 1 \\
3/4 &3 &3 : 1 \\
1 &1/$0\approx\infty$ &$\infty$ \\
\end{tabular}
\end{table}
\end{frame}

\begin{frame}{Confidence interval for odds ratio}
\protect\hypertarget{confidence-interval-for-odds-ratio}{}
\small

Let \(a\), \(b\), \(c\), and \(d\) represent the four cell counts in a
\(2 \times 2\) table.

\[\text{SE}_{\log(\widehat{OR})} = \sqrt{\frac{1}{a} + \frac{1}{b} + \frac{1}{c} + \frac{1}{d}}\]

A \(100(1-\alpha)\)\% confidence interval for
log(OR)\footnote{This CI is valid when all expected cell counts $\geq 10$.}
is given by
\[\log(\widehat{OR}) \pm \left( z^\star \times  \text{SE}_{\text{log($\widehat{OR}$)}} \right) \]

To obtain the confidence interval for OR, exponentiate the bounds of the
CI for log(OR).
\end{frame}

\begin{frame}[fragile]{Confidence interval for odds ratio\ldots{}}
\protect\hypertarget{confidence-interval-for-odds-ratio-1}{}
\small

Compute a 95\% CI for the odds ratio of virologic failure comparing NPV
to LPV.

\[\text{SE}_{\log(\widehat{OR})} = \sqrt{\frac{1}{a} + \frac{1}{b} + \frac{1}{c} + \frac{1}{d}} = \sqrt{\frac{1}{87} + \frac{1}{60} + \frac{1}{113} + \frac{1}{27}} = 0.272\]

95\% CI for log(OR):

\[\log(2.89) \pm (1.96)(0.272) \rightarrow (0.578, 1.595) \]

95\% CI for OR:

\[(e^{0.578}, e^{1.595}) \rightarrow (1.693, 4.920)\]

\scriptsize

\begin{Shaded}
\begin{Highlighting}[]
\FunctionTok{oddsratio}\NormalTok{(hiv.table, }\AttributeTok{rev =} \StringTok{"rows"}\NormalTok{, }\AttributeTok{method =} \StringTok{"wald"}\NormalTok{)}\SpecialCharTok{$}\NormalTok{measure}
\end{Highlighting}
\end{Shaded}

\begin{verbatim}
##      odds ratio with 95% C.I.
## Drug  estimate    lower    upper
##   LPV 1.000000       NA       NA
##   NVP 2.886335 1.693248 4.920088
\end{verbatim}
\end{frame}

\begin{frame}{Relative risk versus odds ratio}
\protect\hypertarget{relative-risk-versus-odds-ratio}{}
\small

The relative risk cannot be used in studies that use
\textbf{outcome-dependent sampling}, such as a case-control study:

\begin{itemize}
\item
  Suppose in the HIV study, researchers had identified 100 HIV-positive
  infants who had experienced virologic failure (cases) and 100 who had
  stable disease (controls), then recorded the number in each group who
  had been treated with NVP or LPV.
\item
  With this design, the sample proportion of infants with virologic
  failure no longer estimates the population proportion.

  \begin{itemize}
  \tightlist
  \item
    Similarly, the sample proportion of infants with virologic failure
    in a treatment group no longer estimates the proportion of infants
    who would experience virologic failure in a hypothetical population
    treated with that drug.
  \end{itemize}
\end{itemize}

The odds ratio remains valid even when it is not possible to estimate
incidence of an outcome from sample data.
\end{frame}

\end{document}
