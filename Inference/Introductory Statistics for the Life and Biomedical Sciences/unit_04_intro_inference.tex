% Options for packages loaded elsewhere
\PassOptionsToPackage{unicode}{hyperref}
\PassOptionsToPackage{hyphens}{url}
%
\documentclass[
  ignorenonframetext,
  aspectratio=169]{beamer}
\usepackage{pgfpages}
\setbeamertemplate{caption}[numbered]
\setbeamertemplate{caption label separator}{: }
\setbeamercolor{caption name}{fg=normal text.fg}
\beamertemplatenavigationsymbolsempty
% Prevent slide breaks in the middle of a paragraph
\widowpenalties 1 10000
\raggedbottom
\setbeamertemplate{part page}{
  \centering
  \begin{beamercolorbox}[sep=16pt,center]{part title}
    \usebeamerfont{part title}\insertpart\par
  \end{beamercolorbox}
}
\setbeamertemplate{section page}{
  \centering
  \begin{beamercolorbox}[sep=12pt,center]{part title}
    \usebeamerfont{section title}\insertsection\par
  \end{beamercolorbox}
}
\setbeamertemplate{subsection page}{
  \centering
  \begin{beamercolorbox}[sep=8pt,center]{part title}
    \usebeamerfont{subsection title}\insertsubsection\par
  \end{beamercolorbox}
}
\AtBeginPart{
  \frame{\partpage}
}
\AtBeginSection{
  \ifbibliography
  \else
    \frame{\sectionpage}
  \fi
}
\AtBeginSubsection{
  \frame{\subsectionpage}
}
\usepackage{amsmath,amssymb}
\usepackage{iftex}
\ifPDFTeX
  \usepackage[T1]{fontenc}
  \usepackage[utf8]{inputenc}
  \usepackage{textcomp} % provide euro and other symbols
\else % if luatex or xetex
  \usepackage{unicode-math} % this also loads fontspec
  \defaultfontfeatures{Scale=MatchLowercase}
  \defaultfontfeatures[\rmfamily]{Ligatures=TeX,Scale=1}
\fi
\usepackage{lmodern}
\ifPDFTeX\else
  % xetex/luatex font selection
\fi
% Use upquote if available, for straight quotes in verbatim environments
\IfFileExists{upquote.sty}{\usepackage{upquote}}{}
\IfFileExists{microtype.sty}{% use microtype if available
  \usepackage[]{microtype}
  \UseMicrotypeSet[protrusion]{basicmath} % disable protrusion for tt fonts
}{}
\makeatletter
\@ifundefined{KOMAClassName}{% if non-KOMA class
  \IfFileExists{parskip.sty}{%
    \usepackage{parskip}
  }{% else
    \setlength{\parindent}{0pt}
    \setlength{\parskip}{6pt plus 2pt minus 1pt}}
}{% if KOMA class
  \KOMAoptions{parskip=half}}
\makeatother
\usepackage{xcolor}
\newif\ifbibliography
\usepackage{color}
\usepackage{fancyvrb}
\newcommand{\VerbBar}{|}
\newcommand{\VERB}{\Verb[commandchars=\\\{\}]}
\DefineVerbatimEnvironment{Highlighting}{Verbatim}{commandchars=\\\{\}}
% Add ',fontsize=\small' for more characters per line
\usepackage{framed}
\definecolor{shadecolor}{RGB}{248,248,248}
\newenvironment{Shaded}{\begin{snugshade}}{\end{snugshade}}
\newcommand{\AlertTok}[1]{\textcolor[rgb]{0.94,0.16,0.16}{#1}}
\newcommand{\AnnotationTok}[1]{\textcolor[rgb]{0.56,0.35,0.01}{\textbf{\textit{#1}}}}
\newcommand{\AttributeTok}[1]{\textcolor[rgb]{0.13,0.29,0.53}{#1}}
\newcommand{\BaseNTok}[1]{\textcolor[rgb]{0.00,0.00,0.81}{#1}}
\newcommand{\BuiltInTok}[1]{#1}
\newcommand{\CharTok}[1]{\textcolor[rgb]{0.31,0.60,0.02}{#1}}
\newcommand{\CommentTok}[1]{\textcolor[rgb]{0.56,0.35,0.01}{\textit{#1}}}
\newcommand{\CommentVarTok}[1]{\textcolor[rgb]{0.56,0.35,0.01}{\textbf{\textit{#1}}}}
\newcommand{\ConstantTok}[1]{\textcolor[rgb]{0.56,0.35,0.01}{#1}}
\newcommand{\ControlFlowTok}[1]{\textcolor[rgb]{0.13,0.29,0.53}{\textbf{#1}}}
\newcommand{\DataTypeTok}[1]{\textcolor[rgb]{0.13,0.29,0.53}{#1}}
\newcommand{\DecValTok}[1]{\textcolor[rgb]{0.00,0.00,0.81}{#1}}
\newcommand{\DocumentationTok}[1]{\textcolor[rgb]{0.56,0.35,0.01}{\textbf{\textit{#1}}}}
\newcommand{\ErrorTok}[1]{\textcolor[rgb]{0.64,0.00,0.00}{\textbf{#1}}}
\newcommand{\ExtensionTok}[1]{#1}
\newcommand{\FloatTok}[1]{\textcolor[rgb]{0.00,0.00,0.81}{#1}}
\newcommand{\FunctionTok}[1]{\textcolor[rgb]{0.13,0.29,0.53}{\textbf{#1}}}
\newcommand{\ImportTok}[1]{#1}
\newcommand{\InformationTok}[1]{\textcolor[rgb]{0.56,0.35,0.01}{\textbf{\textit{#1}}}}
\newcommand{\KeywordTok}[1]{\textcolor[rgb]{0.13,0.29,0.53}{\textbf{#1}}}
\newcommand{\NormalTok}[1]{#1}
\newcommand{\OperatorTok}[1]{\textcolor[rgb]{0.81,0.36,0.00}{\textbf{#1}}}
\newcommand{\OtherTok}[1]{\textcolor[rgb]{0.56,0.35,0.01}{#1}}
\newcommand{\PreprocessorTok}[1]{\textcolor[rgb]{0.56,0.35,0.01}{\textit{#1}}}
\newcommand{\RegionMarkerTok}[1]{#1}
\newcommand{\SpecialCharTok}[1]{\textcolor[rgb]{0.81,0.36,0.00}{\textbf{#1}}}
\newcommand{\SpecialStringTok}[1]{\textcolor[rgb]{0.31,0.60,0.02}{#1}}
\newcommand{\StringTok}[1]{\textcolor[rgb]{0.31,0.60,0.02}{#1}}
\newcommand{\VariableTok}[1]{\textcolor[rgb]{0.00,0.00,0.00}{#1}}
\newcommand{\VerbatimStringTok}[1]{\textcolor[rgb]{0.31,0.60,0.02}{#1}}
\newcommand{\WarningTok}[1]{\textcolor[rgb]{0.56,0.35,0.01}{\textbf{\textit{#1}}}}
\usepackage{graphicx}
\makeatletter
\def\maxwidth{\ifdim\Gin@nat@width>\linewidth\linewidth\else\Gin@nat@width\fi}
\def\maxheight{\ifdim\Gin@nat@height>\textheight\textheight\else\Gin@nat@height\fi}
\makeatother
% Scale images if necessary, so that they will not overflow the page
% margins by default, and it is still possible to overwrite the defaults
% using explicit options in \includegraphics[width, height, ...]{}
\setkeys{Gin}{width=\maxwidth,height=\maxheight,keepaspectratio}
% Set default figure placement to htbp
\makeatletter
\def\fps@figure{htbp}
\makeatother
\setlength{\emergencystretch}{3em} % prevent overfull lines
\providecommand{\tightlist}{%
  \setlength{\itemsep}{0pt}\setlength{\parskip}{0pt}}
\setcounter{secnumdepth}{-\maxdimen} % remove section numbering
\usepackage{amsmath,verbatim}

\usepackage{multirow}
\usepackage{fancyvrb}
\usepackage{manfnt}
\usepackage[normalem]{ulem}

\usepackage{hyperref}
\hypersetup{
	colorlinks=true,
	linkcolor=blue,
	urlcolor=blue}

%\usepackage[colorlinks=true]{hyperref}

\mode<presentation>{\usetheme{Malmoe}}

%\synctex=1

\setbeamertemplate{headline}{}


\setbeamerfont{footline}{size=\scriptsize}
\setbeamerfont{frametitle}{shape=\scshape}
\setbeamertemplate{itemize item}[circle]
\setbeamertemplate{itemize subitem}{\scriptsize$\diamond$}
\setbeamercovered{transparent}

\setbeamertemplate{navigation symbols}{}
\setbeamertemplate{footline}[frame number]{} 


\definecolor{forest}{rgb}{0, .5, 0}
\definecolor{brick}{rgb}{.5, 0, 0}
\definecolor{darkgreen}{rgb}{0, .5, 0}
\definecolor{darkred}{rgb}{.7, .15, .15}
\definecolor{darkblue}{rgb}{0, 0, .5}
\definecolor{Green}{rgb}{0.2,1,0.2}


\newcommand{\R}{\textsf{R}}
\newcommand{\RStudio}{\textsl{R Studio}}

\usepackage[english]{babel}
%\usepackage{palatino}
\usepackage[T1]{fontenc}


% make all tt fonts bold to look more like Verbatim
\usepackage{lmodern}
\renewcommand\ttfamily{\usefont{T1}{lmtt}{m}{n}}

% Comment these out if you don't want a slide with just the
% part/section/subsection/subsubsection title:
\AtBeginPart{
  \let\insertpartnumber\relax
  \let\partname\relax
  \frame{\partpage}
}
\AtBeginSection{
  \let\insertsectionnumber\relax
  \let\sectionname\relax
  \frame{\sectionpage}
}
\AtBeginSubsection{
  \let\insertsubsectionnumber\relax
  \let\subsectionname\relax
  \frame{\subsectionpage}
}

\newcommand{\columnsbegin}{\begin{columns}}
	\newcommand{\columnsend}{\end{columns}}

\newenvironment{twocol}[4]{
\begin{columns}[c]
\column{#1\textwidth}
#3
\column{#2\textwidth}
#4
\end{columns}
}

\def\begincols{\begin{columns}}
\def\begincol{\begin{column}}
\def\endcol{\end{column}}
\def\endcols{\end{columns}}
\ifLuaTeX
  \usepackage{selnolig}  % disable illegal ligatures
\fi
\IfFileExists{bookmark.sty}{\usepackage{bookmark}}{\usepackage{hyperref}}
\IfFileExists{xurl.sty}{\usepackage{xurl}}{} % add URL line breaks if available
\urlstyle{same}
\hypersetup{
  pdftitle={Unit 4: Introduction to Inference},
  pdfauthor={Statistics S-100 Teaching Team},
  hidelinks,
  pdfcreator={LaTeX via pandoc}}

\title{Unit 4: Introduction to Inference}
\author{Statistics S-100 Teaching Team}
\date{Summer 2024}

\begin{document}
\frame{\titlepage}

\begin{frame}[allowframebreaks]
  \tableofcontents[hideallsubsections]
\end{frame}
\hypertarget{sampling-variability}{%
\section{Sampling variability}\label{sampling-variability}}

\begin{frame}{Statistical inference}
\protect\hypertarget{statistical-inference}{}
\small

The goal of \textbf{statistical inference} is to learn about a
population based on the information in a single sample:

\begin{itemize}
\item
  Estimation is the framework for using a \textbf{sample statistic} to
  estimate a \textbf{population parameter}.
\item
  Conducting a \textbf{hypothesis test} allows us to quantify the
  strength of evidence for a specific conjecture made about a population
  parameter.
\end{itemize}

This unit illustrates the principles of inference in the setting of
estimating a population mean.

\begin{itemize}
\item
  Future units will extend inference to other settings.
\item
  We will also discuss important caveats and common misunderstandings in
  future units.
\end{itemize}
\end{frame}

\begin{frame}[fragile]{Youth Risk Factor Behavior Surveillance System
(YRBSS)}
\protect\hypertarget{youth-risk-factor-behavior-surveillance-system-yrbss}{}
\columnsbegin

\column{0.40\textwidth}

\footnotesize

\begin{itemize}
\item
  The YRBSS is a survey conducted by the US CDC to measure
  health-related activity in high school aged youth.
\item
  2.6 million high school students participated between 1991 and 2013.
\item
  The \texttt{yrbss} data in the \texttt{oibiostat} package contains
  responses from 13,583 participants in 2013.
\end{itemize}

\column{0.60\textwidth}

\scriptsize

\begin{Shaded}
\begin{Highlighting}[]
\CommentTok{\#load the data}
\FunctionTok{library}\NormalTok{(oibiostat)}
\FunctionTok{data}\NormalTok{(}\StringTok{"yrbss"}\NormalTok{)}

\NormalTok{yrbss[}\FunctionTok{c}\NormalTok{(}\DecValTok{1}\SpecialCharTok{:}\DecValTok{3}\NormalTok{, }\DecValTok{13582}\NormalTok{, }\DecValTok{13583}\NormalTok{), }\FunctionTok{c}\NormalTok{(}\DecValTok{1}\SpecialCharTok{:}\DecValTok{3}\NormalTok{, }\DecValTok{6}\SpecialCharTok{:}\DecValTok{8}\NormalTok{)]}
\end{Highlighting}
\end{Shaded}

\begin{verbatim}
##       age gender grade height weight   helmet.12m
## 1      14 female     9     NA     NA        never
## 2      14 female     9     NA     NA        never
## 3      15 female     9   1.73  84.37        never
## 13582  17 female    12   1.60  77.11    sometimes
## 13583  17 female    12   1.57  52.16 did not ride
\end{verbatim}

\columnsend
\end{frame}

\begin{frame}[fragile]{Population parameter versus sample statistic}
\protect\hypertarget{population-parameter-versus-sample-statistic}{}
\columnsbegin

\column{0.45\textwidth}

\small

\textbf{Population parameter}

\vspace{0.25cm}

The CDC was interested in estimating the health behaviors of the
\emph{target population}; i.e., all high school students in the US in
2013.

\vspace{0.5cm}

The population mean weight among all high school students in the US in
2013 is an example of a \textbf{population parameter}.

\begin{itemize}
\item
  Typically, the value of a population parameter is unknown.
\item
  The symbol \(\mu\) is used to denote a population mean.
\end{itemize}

\column{0.49\textwidth}

\small

\textbf{Sample statistic}

\vspace{0.25cm}

The descriptive measures for a sample are referred to as \textbf{sample
statistics}.

\vspace{0.3cm}

The sample mean weight among the 13,572 students in \texttt{yrbss} is an
example of a sample statistic.

\begin{itemize}
\item
  The value of a sample statistic is known.
\item
  The symbol \(\overline{x}\) is used to denote a sample mean.
\end{itemize}

\vspace{0.3cm}

The sample mean is a \textbf{point estimate} of the population mean;
i.e., a single value estimate of the population mean.

\columnsend
\end{frame}

\begin{frame}{Sampling variability}
\protect\hypertarget{sampling-variability-1}{}
\small

In nearly all studies, there is one target population and one sample.

\begin{itemize}
\tightlist
\item
  The CDC took a random sample of 13,572 high school students from the
  population of 21.2 million high school students in the US.
\end{itemize}

Suppose that a different random sample (of the same size) were taken
from the same population:

\begin{itemize}
\item
  The sample could consist of different participants, thus\ldots{}
\item
  The sampled weight values could be different, producing a different
  sample mean weight than in the initial sample.
\end{itemize}

\textbf{Sampling variability} refers to the idea that the value of an
estimate (i.e., a sample statistic) varies from sample to sample.

\begin{itemize}
\item
  The value of a sample statistic is not fixed until a specific sample
  is observed.
\item
  The sample mean \(\overline{X}\) is a random variable. When we observe
  a specific sample mean \(\overline{x}\), we are observing one possible
  value that \(\overline{X}\) can take on.
\end{itemize}
\end{frame}

\begin{frame}[fragile]{Sampling from a population}
\protect\hypertarget{sampling-from-a-population}{}
\small

Typically, the exact values of population parameters are
unknown.\footnote{Hence, the need for inference...}

But, we can observe the effect of sampling variability in a
\textbf{hypothetical} setting where \(\mu\) is known. This hypothetical
setting allows us to understand how the distribution of \(\overline{X}\)
behaves.

\begin{itemize}
\item
  Suppose that the 13,572 individuals in \texttt{yrbss} represent our
  target population.

  \begin{itemize}
  \tightlist
  \item
    Let mean weight in \texttt{yrbss} be the population parameter,
    \(\mu_{weight}\).
  \end{itemize}
\item
  Take a random sample from \texttt{yrbss} (e.g., \(n = 30\)) and
  calculate \(\overline{x}_{weight}\), the mean weight among the sampled
  individuals.

  \begin{itemize}
  \tightlist
  \item
    How well does \(\overline{x}_{weight}\) estimate \(\mu_{weight}\)?
  \end{itemize}
\item
  Repeatedly take a random sample and compute the sample mean in order
  to construct a \textbf{sampling distribution of \(\overline{X}\)}.
\end{itemize}
\end{frame}

\begin{frame}[fragile]{Taking one random sample of size 30 from
\texttt{yrbss}}
\protect\hypertarget{taking-one-random-sample-of-size-30-from-yrbss}{}
\footnotesize

\begin{Shaded}
\begin{Highlighting}[]
\CommentTok{\#load the dataset}
\FunctionTok{library}\NormalTok{(oibiostat)}
\FunctionTok{data}\NormalTok{(}\StringTok{"yrbss"}\NormalTok{)}
\end{Highlighting}
\end{Shaded}

\begin{Shaded}
\begin{Highlighting}[]
\CommentTok{\#remove rows with missing values}
\NormalTok{yrbss\_complete }\OtherTok{\textless{}{-}}\NormalTok{ yrbss[}\FunctionTok{complete.cases}\NormalTok{(yrbss}\SpecialCharTok{$}\NormalTok{weight), ]}

\CommentTok{\#set parameters}
\NormalTok{sample\_size }\OtherTok{\textless{}{-}} \DecValTok{30}

\CommentTok{\#obtain random sample of row numbers}
\FunctionTok{set.seed}\NormalTok{(}\DecValTok{5011}\NormalTok{) }
\NormalTok{sample\_rows }\OtherTok{\textless{}{-}} \FunctionTok{sample}\NormalTok{(}\DecValTok{1}\SpecialCharTok{:}\FunctionTok{nrow}\NormalTok{(yrbss\_complete), sample\_size)}

\CommentTok{\#calculate point estimates from sampled rows}
\FunctionTok{mean}\NormalTok{(yrbss\_complete}\SpecialCharTok{$}\NormalTok{weight[sample\_rows]); }\FunctionTok{sd}\NormalTok{(yrbss\_complete}\SpecialCharTok{$}\NormalTok{weight[sample\_rows])}
\end{Highlighting}
\end{Shaded}

\begin{verbatim}
## [1] 69.628
\end{verbatim}

\begin{verbatim}
## [1] 18.26424
\end{verbatim}

\normalsize
\end{frame}

\begin{frame}{The sample mean as a random variable}
\protect\hypertarget{the-sample-mean-as-a-random-variable}{}
\small

Any sample statistic is a random variable, due to the nature of random
sampling from a
population.\footnote{Some additional formalism: Each observation $X_1, X_2, ..., X_n$ in a sample of size $n$ can be thought of as a random variable. A function of random variables is itself a random variable, hence why $\overline{X}$ is a random variable.}

\begin{itemize}
\item
  The value of a sample statistic varies from sample to sample.
\item
  The value of a sample statistic is not fixed until a specific sample
  is observed.
\end{itemize}
\end{frame}

\begin{frame}{The sample mean as a random variable\ldots{}}
\protect\hypertarget{the-sample-mean-as-a-random-variable-1}{}
\columnsbegin

\column{0.55\textwidth}

\small

The \textbf{sampling distribution of $\overline{X}$} conveys the
sample-to-sample variability in \(\overline{X}\).

\begin{itemize}
    
  \small

  \item The sampling distribution of $\overline{X}$ is centered at $\mu$; i.e., $E(\overline{X}) = \mu$.
    
  \item The standard deviation of the sampling distribution, SD$(\overline{X})$, is referred to as the \textbf{standard error of the sample statistic}.
    
  \item The standard error is influenced by:
    
  \begin{itemize}
    
    \item the underlying variation of the data
        
    \item the sample size $n$

\end{itemize}

\end{itemize}

\scriptsize

\column{0.40\textwidth}

\scriptsize

\includegraphics{unit_04_intro_inference_files/figure-beamer/unnamed-chunk-3-1.pdf}

\columnsend
\end{frame}

\begin{frame}{Sampling distribution of a sample statistic\ldots{}}
\protect\hypertarget{sampling-distribution-of-a-sample-statistic}{}
\includegraphics{unit_04_intro_inference_files/figure-beamer/unnamed-chunk-4-1.pdf}

\footnotesize

As \(n\) increases, the sampling distribution becomes more symmetric and
bell-shaped, in addition to becoming more narrow (i.e., the variability
decreases).
\end{frame}

\begin{frame}{Central Limit Theorem (CLT) for the sample mean}
\protect\hypertarget{central-limit-theorem-clt-for-the-sample-mean}{}
\small

If a sufficiently large sample of \(n\) independent observations are
collected from a population with mean \(\mu\) and standard deviation
\(\sigma\), the sampling distribution of \(\overline{X}\) is well
approximated by a normal distribution with the following parameters:
\[E(\overline{X}) = \mu \qquad \text{SD}(\overline{X}) = \frac{\sigma}{\sqrt{n}} \]

\begin{itemize}
\item
  The underlying distribution of individual observations does not need
  to be normally distributed for the CLT to apply, but the distribution
  should not be strongly skewed.
\item
  One rule of thumb: the sample size should be at least \(n = 30\).
\item
  Larger sample sizes are recommended if it is known that the underlying
  distribution of individual observations exhibits skew.
\item
  By the Law of Large Numbers (LLN), \(E(\overline{X}) \rightarrow \mu\)
  as \(n \rightarrow \infty\).
\end{itemize}
\end{frame}

\begin{frame}{A note about handling missingness\ldots{}}
\protect\hypertarget{a-note-about-handling-missingness}{}
\small

Earlier, we based our analysis only on individuals with data for the
\texttt{weight} variable.

\begin{itemize}
\item
  This is an example of doing a \textbf{complete case analysis (CCA)},
  an analysis restricted to individuals with complete information.
\item
  Our conclusions were based on information from 12,579 cases out of the
  13,583 survey participants. The rate of missingness for
  \texttt{weight} was 7.39\%.
\item
  Why might it be important to be transparent about how we handled the
  missing data?
\end{itemize}

\textbf{Imputation} is a common approach for dealing with missingness
that involves using information on other variables to fill in the
missing values with plausible ones. The details are beyond the level of
this
course.\footnote{Check out \textcolor{blue}{\href{https://www.bmj.com/content/338/bmj.b2393}{this article from the BMJ}} if you are interested in learning more about the topic.}
\end{frame}

\hypertarget{confidence-intervals}{%
\section{Confidence intervals}\label{confidence-intervals}}

\begin{frame}{Confidence intervals}
\protect\hypertarget{confidence-intervals-1}{}
\small

A \textbf{confidence interval} provides an estimate for a population
parameter along with a margin of error that gives a plausible range of
values for the population parameter.

A confidence interval for a population mean \(\mu\) has the general form
\[\overline{x} \pm m \rightarrow (\overline{x} - m, \overline{x} + m), \]
where \(m\) is the margin of error.

To calculate \(m\), we use what is known about the \emph{sampling
variability} of \(\overline{X}\).
\end{frame}

\begin{frame}{Using \(s\) as an approximation of \(\sigma\)}
\protect\hypertarget{using-s-as-an-approximation-of-sigma}{}
\small

The \textbf{standard error} of the sample mean,
\(\text{SE}(\overline{X})\), measures the sample-to-sample variability
of \(\overline{X}\); i.e., the extent to which sample means oscillate
around the population mean.

\begin{itemize}
\item
  From theory, the standard error equals \(\dfrac{\sigma}{\sqrt{n}}\).
\item
  In practice, however, \(\sigma\) is typically unknown.
\item
  The sample standard deviation \(s\) is a reasonably good estimate of
  \(\sigma\).
\end{itemize}

Thus, \(\text{SE}(\overline{X})\) is approximately equal to
\(\dfrac{s}{\sqrt{n}}\).

\begin{itemize}
\item
  Using \(s\) to estimate \(\sigma\) introduces another source of
  variability.
\item
  Can adjust for this by modeling the sampling distribution of
  \(\overline{X}\) with the \(t\) distribution rather than the normal.
\end{itemize}
\end{frame}

\begin{frame}{The \(t\) Distribution}
\protect\hypertarget{the-t-distribution}{}
\columnsbegin

\column{0.55\textwidth}

\footnotesize

The \(t\) distribution is symmetric, bell-shaped, and centered at 0.

\begin{itemize}
\item
  Its shape is very close to the (standard) normal distribution, but the
  tails of a \(t\) distribution are thicker. This adjusts for the
  variability introduced by using \(s\) as an estimate of \(\sigma\).
\item
  It has one parameter, \emph{degrees of freedom
  (df)}.\footnote{Degrees of freedom can be represented as $\nu$.}
\item
  When \(df\) is large (\(df \geq 30\)), the \(t\) and \(z\)
  distributions are virtually identical.
\end{itemize}

\column{0.45\textwidth}

\centering

\includegraphics{./figures/tDistConvergeToNormalDist.pdf}

\columnsend
\end{frame}

\begin{frame}{General Form for a confidence interval}
\protect\hypertarget{general-form-for-a-confidence-interval}{}
\small

A \((1 - \alpha)(100)\)\% CI for \(\mu\) is given by
\[\overline{x} \pm t^\star \times \dfrac{s}{\sqrt{n}}, \] where
\(t^\star\), the critical \(t\)-value, is the point on a \(t\)
distribution with degrees of freedom \(n-1\) that has area
\((1 - \alpha/2)\) to the left (and area \(\alpha/2\) to the right).

\begin{itemize}
\item
  The \textbf{margin of error} \(t^\star \dfrac{s}{\sqrt{n}}\) consists
  of the \textbf{standard error} \(\dfrac{s}{\sqrt{n}}\) and the
  critical \(t\)-value \(t^\star\).
\item
  The standard error is computed from features of the sample: \(s\) and
  \(n\).
\item
  The critical \(t\)-value \(t^\star\) changes based on the desired
  level of confidence (and sample size).
\end{itemize}

Common confidence levels: 90\%, 95\%, 99\%
\end{frame}

\begin{frame}[fragile]{Calculating the critical \(t\)-value,
\(t^\star\)}
\protect\hypertarget{calculating-the-critical-t-value-tstar}{}
\small

The function \texttt{qt( )} identifies the point on a \(t\) distribution
with \emph{df} degrees of freedom that has area \(p\) to the left.

\begin{itemize}
\item
  For \(t_{df = n-1}\), \texttt{qt(p, df)} calculates \(t\) such that
  \(p = P(T \leq t)\).
\item
  For a 95\% confidence interval, find the critical \(t\)-value such
  that 95\% of the distribution is between \(-t^\star\) and \(t^\star\).
\item
  The critical \(t\)-value for a 95\% confidence interval where
  \(n = 30\) is 2.045.
\end{itemize}

\scriptsize

\begin{Shaded}
\begin{Highlighting}[]
\FunctionTok{qt}\NormalTok{(}\FloatTok{0.975}\NormalTok{, }\AttributeTok{df =} \DecValTok{29}\NormalTok{)}
\end{Highlighting}
\end{Shaded}

\begin{verbatim}
## [1] 2.04523
\end{verbatim}
\end{frame}

\begin{frame}[fragile]{Calculating a confidence interval}
\protect\hypertarget{calculating-a-confidence-interval}{}
\columnsbegin

\column{0.45\textwidth}

\footnotesize

The confidence interval for population mean weight, from the earlier
sample of 30 individuals:

\[\overline{x} \pm t^\star \frac{s}{\sqrt{n}} \]
\[69.628 \pm (2.045)\frac{18.264}{\sqrt{30}}\]
\[(62.81, 76.45) \text{ kg} \]

\vspace{0.5cm}

\textbf{Interpretation}: With 95\% confidence, the interval (62.81,
76.45) kg captures the population mean weight of high school students
(who responded to the YRBSS survey in 2013).

\column{0.55\textwidth}

\scriptsize

\begin{Shaded}
\begin{Highlighting}[]
\CommentTok{\#letting R do the work}
\FunctionTok{t.test}\NormalTok{(yrbss\_complete}\SpecialCharTok{$}\NormalTok{weight[sample\_rows])}\SpecialCharTok{$}\NormalTok{conf.int}
\end{Highlighting}
\end{Shaded}

\begin{verbatim}
## [1] 62.80802 76.44798
## attr(,"conf.level")
## [1] 0.95
\end{verbatim}

\vspace{0.5cm}

\footnotesize

Full explanation of \texttt{t.test}() coming in the labs.

\columnsend
\end{frame}

\begin{frame}{Interpreting a Confidence Interval}
\protect\hypertarget{interpreting-a-confidence-interval}{}
\small

Suppose that from a random sample of 60 American adults, a 95\%
confidence interval for the population mean weight is (160.89, 185.71)
lbs.

It is tempting to conclude \(\mu\) is within the interval (160.89,
185.71) lbs with probability 0.95\ldots{}

\begin{itemize}
\item
  However, it would be \textbf{incorrect} to claim
  \(P(160.89 \leq \mu \leq 185.71) = 0.95\).
\item
  Before the sample is observed, it is valid to state that the
  \emph{random interval}
  \((\overline{X} - t^\star_{0.975} \frac{s}{\sqrt{n}}, \overline{X} + t^\star_{0.975} \frac{s}{\sqrt{n}})\)
  contains \(\mu\) with probability 0.95.
\item
  Once the sample is observed, the interval is a \emph{fixed interval},
  and either does or does not contain \(\mu\).
\end{itemize}

The correct interpretation relies on the theoretical construct of
repeated sampling. If many samples of \(n = 60\) were taken and a CI
calculated for each one, approximately 95\% of the intervals would
contain \(\mu\).

\begin{itemize}
\tightlist
\item
  Intuitive explanation: a CI expresses values of \(\mu\) consistent
  with the observed data.
\end{itemize}

\normalsize
\end{frame}

\begin{frame}{Interpreting a Confidence Interval\ldots{}}
\protect\hypertarget{interpreting-a-confidence-interval-1}{}
\small

Twenty-five samples of size \(n = 60\) were taken from the `artificial'
population, then a 95\% CI for the population mean adult weight
calculated based on each sample. Only 1 of these 25 intervals did not
contain the population mean, \(\mu = 169.7\)
lbs.\footnote{In a realistic setting, $\mu$ is not known. Here, the complete dataset is treated as the population.}

\centering

\includegraphics[width=0.75\textwidth,height=\textheight]{./figures/95PercentConfidenceInterval.pdf}
\end{frame}

\begin{frame}{Hidden assumptions}
\protect\hypertarget{hidden-assumptions}{}
\small

\begin{enumerate}
\item
  The data used to calculate the confidence interval are from a random
  sample taken from the target population.
\item
  While the population mean from the target population is not known, the
  target population is well-defined.
\end{enumerate}

Both conditions are true in this classroom example of sampling from
\texttt{yrbss}, but may be difficult to verify in practice.
\end{frame}

\hypertarget{hypothesis-testing}{%
\section{Hypothesis testing}\label{hypothesis-testing}}

\begin{frame}{Intuition behind hypothesis testing}
\protect\hypertarget{intuition-behind-hypothesis-testing}{}
\small

Let's first think about a simple coin tossing example. Suppose a friend
gives you a coin and asks you to investigate whether it is fair or
biased.

\begin{itemize}
\item
  First, make a hypothesis. Let's assume that the coin is fair and that
  the probability of seeing heads is equal to
  0.50.\footnote{This is an example of a hypothesis test for a single \textit{proportion}, rather than a \textit{mean}. We'll return to this in Unit 9, but the logic of hypothesis testing is the same regardless of the population parameter of interest.}
\item
  Suppose you plan to flip the coin 20 times and record the outcomes.

  \begin{itemize}
  \tightlist
  \item
    If the coin is actually fair, what should the sampling distribution
    of the number of heads (out of 20 tosses) look like?
  \end{itemize}
\item
  You carry out the 20 tosses\ldots{}

  \begin{itemize}
  \item
    Suppose you see 12 heads. Do you think the coin is biased?
  \item
    Suppose you see 19 heads. Do you think the coin is biased?
  \end{itemize}
\end{itemize}
\end{frame}

\begin{frame}{Example: Cherry Blossom Ten Mile Run}
\protect\hypertarget{example-cherry-blossom-ten-mile-run}{}
\small

The Cherry Blossom Ten Mile Run is an annual road race in Washington,
D.C. The race takes place on the first Sunday in April so as to coincide
with the bloom of the 3,000 cherry trees planted as a gift from the
Japanese government in 1912. In recent years about 15,000 amateur
runners participated.

In 2006, the mean time for all runners who finished the race was 93.3
minutes.

Is the typical US runner getting faster or slower over time?

We will address this question using a random sample of 100 runners who
finished the 2017 Cherry Blossom Ten Mile Run.
\end{frame}

\begin{frame}{Example: Cherry Blossom Ten Mile Run\ldots{}}
\protect\hypertarget{example-cherry-blossom-ten-mile-run-1}{}
\small

Let's apply the logic of hypothesis testing:

\begin{itemize}
\item
  First, make a hypothesis about the population parameter of interest.

  \begin{itemize}
  \item
    Hypothesize that the population mean run time in 2017 is equal to
    the 2006 time.
  \item
    \(H_0\): \(\mu = 93.3\) minutes
  \end{itemize}
\item
  If it were true that \(\mu = 93.3\), what would the sampling
  distribution of \(\overline{X}\) look like when \(n = 100\)?

  \begin{itemize}
  \tightlist
  \item
    We can use simulation to help us visualize this
    distribution.\footnote{We'll have to make some assumptions to visualize the null distribution. More on this in the lab...}
  \end{itemize}
\item
  Where does the observed sample mean run time in 2017 fall on the
  sampling distribution?

  \begin{itemize}
  \item
    Does the evidence support the initial hypothesis that \(\mu = 93.3\)
    minutes?
  \item
    Or, does the evidence suggest that the population mean run time in
    2017 was different from 93.3 minutes?
  \end{itemize}
\end{itemize}
\end{frame}

\begin{frame}{Logic of hypothesis testing}
\protect\hypertarget{logic-of-hypothesis-testing}{}
\small

Observations come from either of two competing distributions:

\begin{itemize}
\item
  The \emph{null} distribution: a usual distribution that has been true
  in the past
\item
  The \emph{alternative} distribution: new distribution induced by an
  intervention or a changing condition
\end{itemize}

We conclude that observations come from the null distribution,
unless\ldots{}

\begin{itemize}
\tightlist
\item
  The value of an \textbf{observed statistic} is so extreme that it
  would be unlikely to occur under the null distribution.
\end{itemize}

In other words, we compare the observed statistic and hypothesized null
value to determine which of the following conclusions is more plausible:

\begin{itemize}
\item
  The difference is small enough that the observed data come from the
  null distribution. Any discrepancy is simply due to sampling
  variability.
\item
  The difference is large enough that the observed data do not come from
  the null distribution. The discrepancy seems too large to be solely
  due to sampling variability.
\end{itemize}
\end{frame}

\begin{frame}{Formal approach to hypothesis testing}
\protect\hypertarget{formal-approach-to-hypothesis-testing}{}
\begin{enumerate}
\item
  Formulate null and alternative hypotheses.
\item
  Specify a significance level, \(\alpha\).
\item
  Calculate a test statistic.
\item
  Calculate a \(p\)-value.
\item
  Draw conclusions in the context of the original problem.
\end{enumerate}
\end{frame}

\begin{frame}{1. Null and alternative hypotheses}
\protect\hypertarget{null-and-alternative-hypotheses}{}
\small

Start with a conjecture about the population parameter of interest, then
formulate statistical null and alternative hypotheses.

The \textbf{null hypothesis} (\(H_0\)) posits a value for the population
parameter of interest and represents a claim to be tested.

\begin{itemize}
\item
  \(H_0\) can also be thought of as representing the status quo; i.e.,
  no change from the past.
\item
  \(H_0: \mu = \mu_0\)
\end{itemize}

The \textbf{alternative hypothesis} (\(H_A\)) is an alternative claim
and is often represented by a range of parameter values.

\begin{itemize}
\tightlist
\item
  \(H_A: \mu \neq \mu_0\), or \(H_A: \mu < \mu_0\), or
  \(H_A: \mu > \mu_0\)
\end{itemize}

Generally, an investigator suspects that the null hypothesis is not true
and performs a hypothesis test in order to evaluate the strength of
evidence against the null hypothesis.
\end{frame}

\begin{frame}{1. Null and alternative hypotheses\ldots{}}
\protect\hypertarget{null-and-alternative-hypotheses-1}{}
\small

We are interested in assessing whether run times seem different in 2017
versus 2006.

\begin{itemize}
\item
  \(H_0: \mu = 93.3 \text{ min}\)\footnote{A null hypothesis has the general format $H_0: \mu = \mu_0$. The symbol $\mu_0$ represents the numeric value that $\mu$ is hypothesized to equal under the null. In this case, the mean run time in 2006 is chosen as $\mu_0$.}
\item
  \(H_A: \mu \neq 93.3 \text{ min}\)
\end{itemize}

This form of \(H_A\) is called a two-sided alternative.

\begin{itemize}
\tightlist
\item
  \(H_A: \mu < 93.3 \text{ min}\) would be a one-sided alternative.
\end{itemize}

The choice of one- or two-sided alternative is context-dependent.
\end{frame}

\begin{frame}{2. Specifying a significance level \(\alpha\)}
\protect\hypertarget{specifying-a-significance-level-alpha}{}
\small

The significance level \(\alpha\) can be thought of as the value that
quantifies how rare or unlikely an event must be in order to represent
sufficient evidence against \(H_0\).

\begin{itemize}
\item
  i.e., if the observed statistic occurs with probability smaller than
  \(\alpha\) under the assumption that \(H_0\) is true, then it is
  considered an extreme observation
\item
  Typically, \(\alpha\) is chosen to be a small value such as 0.10,
  0.05, or 0.01.
\item
  Always specify \(\alpha\) \textbf{before} looking at the data!
\end{itemize}

In the context of decision errors, \(\alpha\) is the probability of
making a Type I error.

\begin{itemize}
\tightlist
\item
  Type I error refers to incorrectly rejecting the null
  hypothesis.\footnote{More on decision errors coming in Unit 6.}
\end{itemize}
\end{frame}

\begin{frame}{3. Calculate a test statistic}
\protect\hypertarget{calculate-a-test-statistic}{}
\small

The test statistic measures the discrepancy between the observed data
and what would be expected if the null hypothesis were true.

\begin{itemize}
\item
  When considering the sampling distribution of \(\overline{X}\), how
  many standard deviations is the observed sample mean from the
  hypothesized population mean, \(\mu_0\)?
\item
  Recall that the standard deviation of \(\overline{X}\) is
  well-approximated by \(\frac{s}{\sqrt{n}}\).
\end{itemize}

When testing hypotheses about a population mean, the test statistic is

\[t = \frac{\overline{x} - \mu_0}{s/\sqrt{n}},\]

where the test statistic \(t\) follows a \(t\) distribution with \(n-1\)
degrees of freedom.

\begin{itemize}
\tightlist
\item
  A larger test statistic indicates greater departure from \(\mu_0\) and
  stronger evidence against \(H_0\) (and in favor of \(H_A\)).
\end{itemize}
\end{frame}

\begin{frame}{4. Calculate a \(p\)-value}
\protect\hypertarget{calculate-a-p-value}{}
\small

\textbf{If the null hypothesis were true}, what is the probability that
we would observe a result \textbf{as or more extreme} than the observed
sample value?

\begin{itemize}
\item
  The \(p\)-value is a conditional probability computed \emph{assuming
  that \(H_0\) is true}.
\item
  The \(p\)-value is \textbf{not} the probability that \(H_0\) is true
  (nor is it the probability that \(H_A\) is false).
\end{itemize}

Using what is known about the distribution of the test statistic,
calculate the \(p\)-value associated with the test statistic then
compare it to the significance level \(\alpha\).

\begin{itemize}
\tightlist
\item
  A result is considered unusual if its \(p\)-value is less than
  \(\alpha\).
\end{itemize}
\end{frame}

\begin{frame}{4. Calculate a \(p\)-value\ldots{}}
\protect\hypertarget{calculate-a-p-value-1}{}
\columnsbegin

\column{0.45\textwidth}

\footnotesize

For a two-sided alternative, \(H_A: \mu \neq \mu_0\), the \(p\)-value is
the total area from both tails of the null distribution that are beyond
the absolute value of the test statistic.

\[p\text{-value} = 2 P(T \geq |t|) = P(T \leq -|t|) + P(T \geq |t|)\]

\begin{figure}[]
\includegraphics[]
{figures/pValueTwoSided.pdf}
\end{figure}

\column{0.45\textwidth}

\footnotesize

For a one-sided alternative, the \(p\)-value is the area in the tail of
the null distribution that matches the direction of the alternative.

\vspace{0.1cm}

For \(H_A: \mu > \mu_0\): \[p\text{-value} = P(T \geq t) \]

\centering

\includegraphics[width=0.8\textwidth,height=\textheight]{./figures/pValueOneSidedR.pdf}

\raggedright

For \(H_A: \mu < \mu_0\): \[p\text{-value} = P(T \leq t) \]

\centering

\includegraphics[width=0.8\textwidth,height=\textheight]{./figures/pValueOneSidedL.pdf}

\columnsend
\end{frame}

\begin{frame}[fragile]{4. Calculate a \(p\)-value\ldots{}}
\protect\hypertarget{calculate-a-p-value-2}{}
\scriptsize

\begin{Shaded}
\begin{Highlighting}[]
\FunctionTok{t.test}\NormalTok{(cherry.sample}\SpecialCharTok{$}\NormalTok{net\_sec}\SpecialCharTok{/}\DecValTok{60}\NormalTok{, }\AttributeTok{mu =} \FloatTok{93.3}\NormalTok{, }\AttributeTok{alternative =} \StringTok{"two.sided"}\NormalTok{)}
\end{Highlighting}
\end{Shaded}

\begin{verbatim}
## 
##  One Sample t-test
## 
## data:  cherry.sample$net_sec/60
## t = 3.0486, df = 99, p-value = 0.002949
## alternative hypothesis: true mean is not equal to 93.3
## 95 percent confidence interval:
##   95.10571 101.83796
## sample estimates:
## mean of x 
##  98.47183
\end{verbatim}

\footnotesize

Alternatively, compute the test statistic and use \texttt{pt()}.

\scriptsize

\begin{Shaded}
\begin{Highlighting}[]
\DecValTok{2}\SpecialCharTok{*}\FunctionTok{pt}\NormalTok{(}\FloatTok{3.0486}\NormalTok{, }\AttributeTok{df =} \DecValTok{100} \SpecialCharTok{{-}} \DecValTok{1}\NormalTok{, }\AttributeTok{lower.tail =} \ConstantTok{FALSE}\NormalTok{) }\CommentTok{\#two{-}sided p{-}value}
\end{Highlighting}
\end{Shaded}

\begin{verbatim}
## [1] 0.002948914
\end{verbatim}

\begin{Shaded}
\begin{Highlighting}[]
\FunctionTok{pt}\NormalTok{(}\FloatTok{3.0486}\NormalTok{, }\AttributeTok{df =} \DecValTok{100} \SpecialCharTok{{-}} \DecValTok{1}\NormalTok{, }\AttributeTok{lower.tail =} \ConstantTok{FALSE}\NormalTok{) }\CommentTok{\#p{-}value for HA: mu \textgreater{} 93.3}
\end{Highlighting}
\end{Shaded}

\begin{verbatim}
## [1] 0.001474457
\end{verbatim}
\end{frame}

\begin{frame}{5. Draw conclusions}
\protect\hypertarget{draw-conclusions}{}
\small

The smaller the \(p\)-value, the stronger the evidence against the null
hypothesis.

\begin{itemize}
\item
  If the \(p\)-value is as small or smaller than \(\alpha\), we
  \textbf{reject} the null hypothesis. The result is statistically
  significant at level \(\alpha\) and there is evidence to accept the
  alternative hypothesis.
\item
  If the \(p\)-value is larger than \(\alpha\), we \textbf{fail to
  reject} the null hypothesis. The result is not statistically
  significant at level \(\alpha\). In other words, the evidence does not
  contradict the null hypothesis.
\end{itemize}

A subtle but important point: not rejecting \(H_0\) is not the same as
proving that \(H_0\) is true. We simply do not have sufficient evidence
that \(H_0\) is not true!
\end{frame}

\begin{frame}{5. Draw conclusions\ldots{}}
\protect\hypertarget{draw-conclusions-1}{}
\small

A \(p\)-value represents the probability of observing a test statistic
\textbf{as extreme or more extreme} than what was observed, under the
\textbf{assumption that the null hypothesis is true.}

Interpretation in context:

\begin{itemize}
\item
  Let's use the \(p\)-value for \(H_A: \mu > 93.3\) min.
\item
  The chance of seeing a sample of 100 runners with mean completion time
  of 98.5 minutes or greater if the actual mean completion time for all
  runners in 2017 were actually 93.3 minutes equals 0.0015.
\end{itemize}

Conclusion:

\begin{itemize}
\item
  Since 0.0015 is very small (relative to \(\alpha = 0.05\)), these data
  suggest evidence against the null hypothesis that the mean completion
  time in 2017 is 93.3 minutes. There is evidence at the
  \(\alpha = 0.05\) significance level to reject \(H_0\) and accept
  \(H_A: \mu > 93.3\) min.
\item
  It would be very unusual to see a sample mean of 98.5 minutes or
  greater if the population mean were actually 93.3 minutes, so we can
  conclude that the mean completion time in 2017 is greater than 93.3
  minutes.
\end{itemize}
\end{frame}

\begin{frame}{5. Draw conclusions\ldots{}}
\protect\hypertarget{draw-conclusions-2}{}
\small

What if the sample mean had actually been 94.5 minutes, with a one-sided
\(p\)-value of 0.241?

Conclusion:

\begin{itemize}
\item
  Since 0.241 is not small (relative to \(\alpha = 0.05\)), these data
  \emph{do not suggest evidence against the null hypothesis} that the
  mean completion time in 2017 is 93.3 minutes.
\item
  Instead, these data are consistent with the sample-to-sample variation
  that we would expect when drawing a random sample of 100 completion
  times when the average completion time is actually 93.3 minutes.
\item
  Since \(p > \alpha\), there is insufficient evidence to reject
  \(H_0\); i.e., we fail to reject \(H_0\).
\item
  \textbf{Note}: In statistics, we do not ``accept'' \(H_0\). We simply
  have not seen evidence against \(H_0\).
\end{itemize}
\end{frame}

\begin{frame}{Guidance on interpreting \(p\)-values}
\protect\hypertarget{guidance-on-interpreting-p-values}{}
\small

A \(p\)-value represents the probability of observing results \emph{as
or more extreme than what was observed}, under the \emph{assumption that
the null hypothesis is true}.

\begin{itemize}
\item
  A ``small'' \(p\)-value suggests that the null hypothesis is not true;
  i.e., represents evidence that we can reject the null hypothesis.

  \begin{itemize}
  \item
    Conceptual thinking: We observed results that are so unlikely to
    happen if the null is true that it makes more sense to conclude that
    the null is false.
  \item
    Example: We flip a coin 100 times and see heads 95 times. This is so
    unlikely to happen with a fair coin that it seems reasonable to
    conclude the coin is biased.
  \end{itemize}
\item
  A ``large'' \(p\)-value indicates that there is insufficient evidence
  against the null hypothesis; i.e., the results are consistent with the
  null hypothesis (being true).

  \begin{itemize}
  \item
    This is \textbf{not} the same as claiming that we have found
    evidence that the null is true! We simply have not yet observed
    evidence indicating the null hypothesis is false.
  \item
    Example: We flip a coin 100 times and see heads 45 times. This is
    not that unlikely to happen with a fair coin, so we can't conclude
    the coin is biased. We also have not proven that the coin is fair.
  \end{itemize}
\end{itemize}
\end{frame}

\begin{frame}{Guidance on interpreting \(p\)-values\ldots{}}
\protect\hypertarget{guidance-on-interpreting-p-values-1}{}
\small

What is considered extreme enough evidence to be convincing?

\begin{itemize}
\item
  A common practice is to consider \(p\)-values smaller than 0.05 to be
  ``statistically significant''.

  \begin{itemize}
  \tightlist
  \item
    Intuition: If we see results that would only happen less than 5\% of
    the time when the null hypothesis is true, then we have sufficient
    evidence against the null hypothesis.
  \end{itemize}
\item
  In scientific literature, this benchmark is referred to as \(\alpha\);
  i.e., the significance level.
\item
  The value of \(\alpha\) is pre-specified before an analysis is done.
  Common values of \(\alpha\) include 0.10, 0.05, 0.01, etc.
\end{itemize}

However, we should avoid thinking of \(\alpha = 0.05\) as a definitive
cutoff.

\begin{itemize}
\item
  How should we interpret a \(p\)-value of 0.08?
\item
  How should we interpret a \(p\)-value of 0.045?
\end{itemize}
\end{frame}

\begin{frame}{Assumptions}
\protect\hypertarget{assumptions}{}
\small

The same assumptions behind the CLT are behind hypothesis testing:

\begin{itemize}
\item
  Observations are independent of each other.
\item
  The sample size is reasonably large.
\end{itemize}

The same hidden assumptions behind confidence intervals also apply:

\begin{itemize}
\item
  The data used for the hypothesis test are representative of the target
  population.
\item
  The target population is well-defined.
\end{itemize}
\end{frame}

\begin{frame}{Two-sided vs.~One-sided Hypothesis Tests}
\protect\hypertarget{two-sided-vs.-one-sided-hypothesis-tests}{}
\columnsbegin

\column{0.50\textwidth}

\small

Make this choice \textbf{before} looking at the data!

\vspace{0.25cm}

In practice, two-sided tests are seen as more rigorous and are typically
expected by most journals and regulatory authorities.

\vspace{0.25cm}

It is ``easier'' to reject \(H_0\) for a one-sided test than a two-sided
test, when conducted at the same \(\alpha\)
level.\footnote{More about choosing between one-sided and two-sided tests in \textit{OI Biostat} Section 4.3.5}

\column{0.50\textwidth}

\begin{figure}[]
\includegraphics[]
{figures/twoSidedTestConservative.pdf}
\end{figure}

\columnsend
\end{frame}

\hypertarget{the-relationship-between-tests-and-confidence-intervals}{%
\section{The relationship between tests and confidence
intervals}\label{the-relationship-between-tests-and-confidence-intervals}}

\begin{frame}{Two-sided tests and confidence intervals}
\protect\hypertarget{two-sided-tests-and-confidence-intervals}{}
\small

The relationship between a hypothesis test and the corresponding
confidence interval is defined by the significance level, \(\alpha\).

\begin{itemize}
\item
  Hypothesis test: Is the sample statistic far enough away from the
  hypothesized null value to be considered extreme?
\item
  Confidence interval: Is the hypothesized null value close enough to
  the sample statistic to be plausible?
\end{itemize}

The ``far enough'' and ``close enough'' are defined by \(\alpha\).

\begin{itemize}
\item
  If a 95\% CI for a parameter does not contain the hypothesized null
  value, then the data contradict the null hypothesis at significance
  level \(\alpha = 0.05\).
\item
  If a 95\% CI for a parameter does contain the hypothesized null value,
  then the data do not contradict the null hypothesis at significance
  level \(\alpha = 0.05\).
\end{itemize}
\end{frame}

\begin{frame}[fragile]{Two-sided tests and confidence intervals\ldots{}}
\protect\hypertarget{two-sided-tests-and-confidence-intervals-1}{}
\columnsbegin

\column{0.50\textwidth}

\footnotesize

If a 95\% confidence interval for a population mean does not contain a
hypothesized value \(\mu_0\), then:

\begin{itemize}
\item
  The data contradict the the null hypothesis \(H_0: \mu = \mu_0\) at
  significance level \(\alpha = 0.05\)
\item
  The implied two-sided alternative hypothesis is
  \(H_A: \mu \neq \mu_0\)
\end{itemize}

\column{0.50\textwidth}

\footnotesize

Based on the following 95\% confidence interval computed from the 2017
sample of run times, is there sufficient evidence to reject
\(H_0: \mu = 93.3\) minutes at the \(\alpha = 0.05\) significance level?

\vspace{0.3cm}

\scriptsize

\begin{Shaded}
\begin{Highlighting}[]
\FunctionTok{t.test}\NormalTok{(cherry.sample}\SpecialCharTok{$}\NormalTok{net\_sec}\SpecialCharTok{/}\DecValTok{60}\NormalTok{, }
       \AttributeTok{conf.level =} \FloatTok{0.95}\NormalTok{)}\SpecialCharTok{$}\NormalTok{conf.int}
\end{Highlighting}
\end{Shaded}

\begin{verbatim}
## [1]  95.10571 101.83796
## attr(,"conf.level")
## [1] 0.95
\end{verbatim}

\columnsend
\end{frame}

\begin{frame}{In practice\ldots{}}
\protect\hypertarget{in-practice}{}
\small

A confidence interval provides a range of plausible values for a
parameter.

A hypothesis test measures the strength of evidence against a null
hypothesis.

Both are important tools for conducting inference.

\begin{itemize}
\tightlist
\item
  A lot more information/guidance about interpreting confidence
  intervals and hypothesis tests coming up in future units.
\end{itemize}
\end{frame}

\end{document}
